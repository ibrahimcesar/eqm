\chapter{Novo caminho}

Esta é mais uma história de alguém que cresceu e não soube que fazer quando chegou lá.

Quando se é jovem, você tem uma espécie de potencial infinito. É como se você pudesse fazer o que quiser, ser quem você quiser. Para onde quer que se olhe não parece haver limites.

Você sente que pode fazer tudo, que pode mudar o mundo, a forma como se entende a realidade ou como as pessoas agem. Pode dar a volta ao mundo em menos de oitenta dias, pode cruzar os sete mares, salvar a princesa e conseguir a paz mundial. Você pode ser como qualquer uma das grandes personalidades que mostraram que as ideias e as ações de uma pessoa podem mudar completamente o rumo de milhões de outras. Você pode ser Einstein. Você pode ser Gandhi. Jesus, Hitler, Tesla. John Lennon ou Lady Di.

Mas com o tempo, mais e mais o mundo vai girando e você se percebe no mesmo lugar de antes. Então, um dia, percebe que sem nem ao menos conseguir precisar onde ou como, você perdeu todo aquele potencial infinito.

Você não é nenhum Einstein. Você não é Gandhi. Não é Jesus. Nem Hitler ou Tesla. Você não se tornou um Lennon ou Lady Di. Você não conseguiu mudar o mundo. O mais correto seria dizer que o mundo acabou mudando você. Você não deu a volta ao mundo. Na verdade mal saiu do bairro em que cresceu. Não consegue nadar nem mesmo em uma piscina infantil, para qualquer princesa você é um sapo e, para falar a verdade, está em guerra até hoje com seu irmão.

E tudo o que você se tornou foi\ldots\,Você. E é difícil se contentar com tão pouco. Essa é a sua vida. Cada hora a mais é na verdade uma hora a menos.

Era assim que se sentia Jonas Arcádio da Silva, ou Jota, como também é conhecido. Ele é do tipo de sujeito que se faz perguntas. Todo tipo de perguntas. Desde as mais triviais\ldots

--- Sanduíche de presunto ou atum?

\ldots\,Até as colossais\ldots

--- Qual é o sentido disso tudo, afinal?

Todos nós alguma vez em nossas vidas nos deparamos com escolhas. Com Jonas não foi diferente, mas ele sempre abriu mão de fazer qualquer uma, aceitando o que quer que acontecesse a ele. Agora ele pensava no peso disso e nas implicações que ele facilmente poderia ter previsto. Ele não estava feliz. Estava vivendo uma vida que não escolheu viver. Como foi chegar a uma situações dessas? Ele estava ficando velho. Sua vida chegando ao fim a cada minuto que passa. Cada hora a mais na verdade sendo uma hora a menos. Jonas tinha perfeita consciência disso, mas continuava lá, parado.

Ele não se achava feio. Mas tampouco era bonito. Era inteligente demais para não perceber que esteticamente era pouco atraente. Mas, em comparação, até sujeitos muito feios se davam bem, como Matias, por exemplo. Para se ter uma ideia de quão feio ele era, basta pensar em uma pessoa bem feia que você conheça. Ele era mais. Do tipo que nem mesmo a mãe dele devia dizer que era lindo, como todas as mães fazem. E mesmo Matias arranjava namoradas. ``Qual o problema comigo?'', ele se atormentava com a dúvida.

--- Talvez você seja um Homo --- lhe disse uma vez George.

--- Nem que for por um simples questão estética\mudanca{,} eu prefiro ser heterossexual --- Jonas respondeu.

George deu um sorriso diabólico de quem sabe mais do que diz.

--- Nós todos somos Homo, Jonas. Homo \mudanca{s}apiens. Pergunta: se você \mudanca{pudesse} ser qualquer super-herói, qual você seria?

--- Homem-Aranha --- Jonas disse após pensar por um tempo.

--- Uhm\ldots\,--- George pensou um pouco antes de falar. --- O Homem-Aranha é legal. Mas ele mal consegue pagar o aluguel. Eu escolheria ser o Senhor~Fantástico. O filho dele tem o poder de moldar a realidade. É o poder mutante mais legal. E além disso ele é rico, inteligente e faz sexo com a Jessica Biel.

--- Alba. Jessica Alba --- Jonas o corrigiu.

--- É --- ele falou. Com a Jessica Alba também. Por que escolher quando se pode ter todas?

O telefone tocou. Só podia significar que alguém de outro departamento estava tendo problemas com computadores. Isso é parte do que eles fazem, ajudar as pessoas quando seus computadores dão uma de humanos e falham. Ao invés de fazer um acordo de revezamento no atendimento às chamadas, George dizia que deveriam disputar sempre uma melhor de três no joquempô, pois, dessa forma, segundo ele, tal questão não ficaria a cargo da sorte ou de acordos e sim de quem é o melhor.

Eles se olharam, \mudanca{haveria} um duelo. George abriu um sorriso. Eles bateram suas mãos fechadas contra a palma da mão e contaram três vezes, Jonas colocou papel e George também. Empate. Na segunda, Jonas colocou tesoura e George pedra, soltando um grito de alegria ao perceber que estava a apenas um passo da vitória. Não que Jonas não quisesse atender o telefone, afinal isso era parte de seu emprego, algum deles teria que atender de um jeito ou de outro. Jonas somente não queria perder pois sabia que isso significaria ter George o lembrando todo o tempo que ele é o melhor até eles disputarem outra vez e vencer. Os piores dias de trabalho de Jonas ocorreram quando George ficou invicto por vinte e três vezes. E George jamais iria esquecer uma coisa dessas. Jonas colocou pedra e George papel. George se levantou e ficou comemorando para uma plateia invisível, talvez lotada com todos seus amigos e namoradas invisíveis.

--- Sim! Sim! --- ele comemorava. --- Quem é o melhor? Quem é o melhor? Lembra quando eu venci vinte e três vezes seguidas? Vin-te-e-trê-es ve-ze-es!

Jonas atendeu o telefone logo após ouvi-lo dar outro toque. Enquanto estava com o aparelho, olhava para George com uma expressão séria enquanto ouvia o que alguém estava dizendo do outro lado da linha. Levantou as sobrancelhas e, balançando, ficou com uma cara assustada, colocando o aparelho de volta sem dizer por algum tempo qualquer palavra, criando um clima de suspense na sala.

--- O que aconteceu? --- George perguntou, também assustado.

--- Desligaram antes que eu atendesse --- respondeu Jonas com um sorriso nos lábios.

--- Você é um idiota, Jonas --- falou George voltando sua atenção para o computador.

Jonas nunca conseguia achar nenhuma resposta que o satisfizesse. E assim ele continuava se fazendo perguntas, até mesmo se perguntando por que fazer perguntas. \mudanca{Era} um jovem cheio de seu potencial infinito quando ele arranjou um emprego na Lethe~Ltda. Quando esteve lá para o seu primeiro dia de emprego, estava excitado em aprender coisas novas com a certeza de que aquilo seria apenas algo passageiro, que ele ficaria ali até um dia simplesmente sair e realizar seus sonhos.

Hoje fazia sete anos em seu emprego passageiro. Realizar seus sonhos não estava em sua agenda a curto prazo. Na verdade, Jonas nem mesmo tinha uma agenda. E mesmo que \mudanca{tivesse,} ela estaria cheias de páginas em branco, assim como seria um livro sobre sua vida sexual.

Ele contentava-se em viver sua vida entre a manhã e a hora de ir dormir. Improvisando no meio do caminho. E se fazendo milhões de perguntas sem nunca aceitar nenhuma resposta que simplesmente aparecesse no meio do caminho, como um filósofo inglês disse, eternamente não realizado, e sem saber por quê, como uma busca por pistas de um assassinato nos olhos de um homem morto.

Então, naquele dia em que completava sete anos trabalhando no mesmo lugar, mostrou a George algo que simplesmente poderia mudar o rumo de sua vida, como se ele sempre estivesse esperando por aquilo.

--- Era isso que você queria me mostrar? --- perguntou George. --- Você precisa urgentemente de uma namorada.

--- Você não está entendendo as implicações metafísicas disso --- falou Jonas.

--- E você não está entendendo as implicações físicas disso --- retrucou George. --- Pessoas morrem fazendo isso, Jonas.

Era verdade. O que Jonas mostrava a George era um folheto que ele havia achado caído \mudanca{no caminho}. Ele teve que ir caminhando para o trabalho naquele dia. Não podia usar o carro e o pneu de sua bicicleta estava furado. Normalmente ele nem teria dado atenção, mas quando seus olhos bateram no papel, o Ankh estampado na capa o atraiu.

O Ankh é um símbolo hieróglifo para a morte, mas que também está associado à vida eterna. Foi apropriado para usar como logo da Novo Caminho e seu revolucionário processo. O que eles ofereciam era tão tentador quanto perigoso. Qual é o denominador comum que todas as pessoas de todas épocas e lugares sempre buscaram? Comida, sexo, moradia, impostos baixos talvez, felicidade e um propósito para suas vidas.

O folheto da Novo Caminho tocou Jonas profundamente. Sua vida inteira ele esteve esperando basicamente por duas coisas: uma guerra ou uma namorada. Cansado de sua vida, ele esperava que alguma dessas duas coisas, não importava qual, talvez a que viesse primeiro, simplesmente salvasse a sua vida. A primeira não seria muito difícil, ao menos em sua cabecinha, já que ele tinha uma teoria, na verdade um ``achismo''\mudanca{,} de que um dia, em um futuro bem próximo, o Brasil entraria em guerra por possuir recursos naturais que serão de importância para alguns outros países. Água, por exemplo. Haveria uma guerra por água.

--- Como os americanos matando árabes por petróleo --- ele comparava.

Quanto mais ele pensava em uma namorada\mudanca{, mais tinha} a sensação de que ela estava oculta em algum ponto do futuro não muito próximo. Algumas pessoas simplesmente terminam seus dias sozinhos. Ele tentava se conformar com isso.

Talvez \mudanca{devesse} simplesmente se conformar mesmo. Ele tentava. Tomava a decisão de que deixaria de esperar por isso, mas bastava ir na padaria, por exemplo, passar por uma garota linda na rua e lá estava ele novamente não deixando de pensar naquela pessoa que \mudanca{não} sabia o nome, quem era, de onde veio, para onde foi. Nada. Geralmente acabavam se tornando apenas mais uma fantasia de banheiro.

Ele também, verdade seja dita, não se esforçava muito em prol de seus objetivos. Como lidava muito mal com a rejeição, tendo poucas ainda que épicas, simplesmente evitava a rejeição por não tentar.

Seu tipo ideal de encontro seria a garota dando em cima dele. O problema desse cenário hipotético é que ele provavelmente ficaria assustado demais para falar qualquer coisa, e garotas que dão em cima de rapazes geralmente não \mudanca{faziam} o ``tipo'' de garota pela qual ele tem preferência. Aqui, outro problema.

Mesmo sabendo de suas limitações, estéticas e sociais, e seu anseio por ter uma namorada, ainda assim ele tinha algumas restrições e exigências que deveria ser preenchidas para o cargo. Não deixava de ser um paradoxo. Isso, é claro, não quer dizer nada. A vida é um paradoxo.

Somos organismos vivos feitos de carbono e outros elementos químicos que somados correspondem a cerca de cinco por cento de toda a matéria presente no Universo. Viajando pelo espaço em um pálido ponto azul ao redor de uma estrela chamada Sol, e apenas por que possuem um cérebro mais desenvolvido que os outros organismos da terceira rocha a partir da estrela considera-se o centro do Universo. Existe paradoxo maior que esse? Depois do Big Bang nada mais é impossível. Nem mesmo morrer e voltar de lá. Quando isso acontece\mudanca{,} chamam de \textsc{eqm}. E a novidade é que não demorou muito até que alguém tivesse a ideia de explorar isso comercialmente, como organizar viagens à Disney.

\textsc{eqm} é \mudanca{a} sigla para Experiências de Quase-Morte. Uma \textsc{eqm} é, por definição, uma experiência subjetiva distinta que algumas pessoas relatam em episódios em que a pessoa foi declarada morta, próxima da morte ou em estados onde a morte seria esperada. Duas experiências de quase-morte nunca são idênticas. O tipo mais comum envolve sentimentos intensos de paz, prazer e amor. A mais antiga delas se encontra no ``Mito de Er'' no final do décimo livro que compõe \emph{A república} de Platão, escrito por volta de 420~a.C.

Neste mesmo livro, Platão apresenta um dos conceitos \mudanca{talvez} mais conhecidos de toda a filosofia\mudanca{:} o mito da caverna. A maioria das pessoas sabe de cor. Imaginemos um muro alto separando o mundo externo \mudanca{de} uma caverna. Na caverna existe uma fresta por onde passa um feixe de luz exterior. No interior\mudanca{,} permanecem seres humanos, que nasceram e cresceram ali.

Ficam de costas para a entrada, acorrentados, sem poder locomover-se, forçados a olhar somente a parede do fundo da caverna, onde são projetadas sombras de outros homens que, além do muro, mantêm acesa uma fogueira. Os prisioneiros julgam que essas sombras \mudanca{são} a realidade.

Um dos prisioneiros decide abandonar essa condição e fabrica um instrumento\mudanca{,} com o qual quebra os grilhões. Aos poucos vai se movendo e avança na direção do muro e o escala, com dificuldade enfrenta os obstáculos que encontra e sai da caverna, descobrindo não apenas que as sombras eram feitas por homens como eles, e mais além todo o mundo e a natureza.

Pessoas que passam por experiências de quase-morte relatam o mesmo que o sujeito que saiu da caverna e viu um outro mundo lá fora. Isso muda suas vidas. Era tudo o que Jonas queria. Mudar sua vida, chacoalhar seu cotidiano. Na falta de uma guerra e de uma namorada era exatamente o que ele procurava.

E se você soubesse que existe uma espécie de clínica, chamada Novo Caminho, que oferece um processo não-ci\-rúr\-gi\-co para que \mudanca{tivesse} voluntariamente uma experiência de quase-morte? Jonas descobriu isso quando apanhou o folheto com um Ankh \mudanca{estampado}. Mas\mudanca{, mesmo assim,} ele se sentia dominado pelo medo e hesitante.

--- Você nunca quis saber o que existe do outro lado? --- perguntou Jonas em um tom empolgado.

--- Não --- disse George, como se a resposta fosse a mais óbvia de todas. --- Eu odeio \emph{spoilers}.

Jonas ficou olhando para o folheto. ``Será que tenho coragem para fazer isso?'', pensou sentindo frio em seu estômago. Colocou o folheto em uma gaveta e voltou a trabalhar.

Esta é a sua vida. Cada hora a mais é na verdade uma hora a menos.

Meu nome é Jonas. Estou escrevendo na terceira pessoa apenas para alcançar a tão almejada imparcialidade.
