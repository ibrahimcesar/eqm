\chapter{Sarah}

Iria acontecer mais cedo ou mais tarde, disso Sarah não tinha dúvidas. Mas por que tinha que acontecer justo hoje e bem no meio de uma aula de matemática\mudanca{, c}om aquele professor tão austero e distante dos alunos\mudanca{?} Primeiro ela sentiu umas pontadas em sua barriga. Achou que \mudanca{era} algo que \mudanca{havia comido}. Mas então ela sentiu. Menarca. A primeira menstruação de um mulher. Ou garota, já que ela só tinha treze anos.

Ela levantou a mão\mudanca{: o} professor colocava em ritmo frenético equações no quadro-negro e ela não sabia se devia incomodá-lo.

--- Professor? --- ela disse com receio.

--- Sim? --- ele disse\mudanca{, virando-se}.

Ele ficou \mudanca{olhando} por um segundo ou dois, esperando que ela fizesse alguma pergunta.

--- Pode dizer, Sarah --- ele disse\mudanca{,} mexendo na armação de seus óculos.

Ela ficou em silêncio.

--- Por favor\ldots\,--- ela disse, sabendo que todos os vinte e três pares de olhos da sala estavam sobre ela.

O professor se aproximou dela, sem saber o que a menina queria. Ela se aproximou dele e sussurrou. Pronunciou cada palavra com cuidado e medo. Vergonha também.

--- Oh --- ele disse\mudanca{,} olhando para ela sem ter ideia alguma do que fazer. --- Problemas de garotas.

Ele chamou a inspetora de alunos. Ela guardou seu material. Foi ao banheiro da escola para saber quais haviam sido os estragos. Ela estava tão petrificada de vergonha que mal conseguia andar. Jogou água em seu rosto. E se olhou no espelho. Doía como se houvesse alguma espécie de gato se agarrando em sua barriga. Cólica.

``Droga. Droga'', ela pensava com uma mão na barriga e outra apoiada na pia. ``Eu nem mesmo quero ter filhos''.

Quando ela saiu, a inspetora lhe perguntou se estava tudo bem.

--- Acho que dá para esperar meu pai chegar.

--- Acontece --- a inspetora disse em tom confidente.

Então ela esperou em pé até que \mudanca{seu pai} chegasse. Ele pegou sua bolsa e não disse nenhum palavra até entrarem no carro. \mudanca{Estava} meio atrapalhado, como se estivesse ensaiando várias vezes o que dizer \mudanca{falar} nenhuma bobagem.

--- As abelhinhas\ldots\,--- ele começou.

--- Preciso de absorventes --- Sarah disse.

--- Oh, sim. Claro --- ele disse\mudanca{,} ligando o carro. --- Queria que sua mãe estivesse aqui.

Sarah também.

Ela se lembra dos longos banhos de banheira que ela e sua mãe tinham. Sua mãe \mudanca{lhe contava} de todos os livros que havia lido, de sua viagem \mudanca{à} Europa e de como ela conhecera seu pai. Ela queria ter lembranças mais sólidas sobre esse tempo de sua vida, mas eram apenas reflexos de quase-memórias, estilhaços de bons momentos que ela não saberia diferenciar de um sonho.

Mas havia nela uma lembrança que era mais real que qualquer outra. Quando ela sentiu o carro sendo levado e \mudanca{lascas} de vidros voarem ela soube, \mudanca{teve} a certeza, de que \mudanca{jamais} se esqueceria daquilo tudo. Na verdade\mudanca{,} ela sabia que ia lembrar daquilo tudo um milhão de outras vezes. Uma memória que toca sem parar, em \emph{replay} automático. Um \emph{loop} sentimental.

Quando \mudanca{chamou seu pai}, sentido \mudanca{dor} em sua cabeça e não teve resposta, \mudanca{sentiu} medo. Caía um pó branco que tomou conta do carro. Eram do \emph{airbag}. Ela não conseguia se mexer.

Aquele dia não tinha nada de especial. Seu pai iria voltar de um congresso em outra cidade, mais um entre \mudanca{tantos} outros. Sua mãe passara o dia pintando um quadro. Ela rindo \mudanca{à} toa com todos aqueles personagens de desenho animado. Eram tão legais.

A viagem de seu pai iria demorar no máximo uma hora de avião. Então sua mãe resolveu que sairiam mais cedo para comprar algo no supermercado e fazer um jantar especial para ele.

--- Posso me sentar no banco da frente? --- Sarah perguntou.

--- Só quando você tiver dez, querida.

--- Por favor, por favor!

Sua mãe ficou olhando para seu pequeno bebê\mudanca{,} que já estava se tornando uma garota crescida. O que será que se passaram por seus olhos castanhos? O que Larissa viu naquele momento?

--- Só dessa vez --- Larissa insistiu várias vezes.

Quando estavam saindo com o carro, sua mãe ligou pelo viva-voz para saber se ele já tinha saído de lá.

O telefone tocou uma, duas vezes.

--- Acho que ele já embarcou --- ela falou com Sarah\mudanca{,} que estava entretida olhando para a paisagem. \mudanca{Finalmente} havia conseguido sentar no banco da frente.

--- O que foi? --- uma voz em tom ríspido perguntou do outro lado. Larissa olhou para a filha que ao que lhe parecia não havia notado o tom de voz com que o pai atendera o telefone.

--- Só queria saber se está tudo bem --- falou Larissa\mudanca{,} tentando entender que devia estar sendo um dia ruim para ele. Se ela e Roberto nunca haviam tido grandes discussões\mudanca{,} isso se devia única e exclusivamente por causa de Larissa. Roberto\mudanca{,} por outro lado, parecia se esforçar com zelo particular em brigar. Larissa sempre procurava ser compreensiva. Não que ela não tivesse defeitos. Nem que ela simplesmente se dobrasse aos desejos dele, pelo contrário. Ela apenas sabia ser mais diplomática e menos explosiva. Ela era do tipo que fingiria estar tudo bem e no outro dia você encontraria um bilhete onde ela explicava seus motivos.

--- Estaria se você não ficasse me ligando. Estou atrasado --- Roberto falou rispidamente.

--- Eu te amo querido --- ela disse\mudanca{,} ao mesmo tempo que ecoava no carro o som do telefone sendo desligado. Ela respirou fundo.

--- Vai querer mesmo começar a fazer teatro? --- ela perguntou para Sarah\mudanca{,} tentando pensar em outra coisa.

--- Não sei.

Compraram o vinho caro que ele gostava, aquele queijo fedido e outras coisas que Sarah nem deu-se ao trabalho de ver o que era. Saíram do supermercado em direção ao aeroporto. Ela se lembra de que estava bem no fim de tarde, quando o Sol já se escondeu\mudanca{,} mas ainda não está completamente escuro.

--- Quer saber um segredo? --- Larissa perguntou.

Sarah virou o rosto, havia sido fisgada pela curiosidade. Sua mãe passou uma mão em sua cabeça, sorriu e foi dizer algo\mudanca{,} quando um carro em alta velocidade se chocou com \mudanca{seu lado}. Sarah sentiu o carro sendo levado e estilhaços de vidros voarem. Ela soube, \mudanca{teve} a certeza, de que \mudanca{jamais} se esqueceria daquilo tudo. Na verdade\mudanca{,} sabia que ia lembrar daquilo tudo um milhão de outras vezes. Uma memória que toca sem parar, em \emph{replay} automático.

Quando \mudanca{chamou seu pai}, sentido \mudanca{dor} em sua cabeça e não teve resposta, \mudanca{sentiu} medo. Caía um pó branco que tomou conta do carro. Eram do \emph{airbag}. Ela não conseguia se mexer. \mudanca{Foi} ficando com sono, ouviu alguém gritar lá fora. Suas pálpebras pesando e pesando. Dormir, dormir. Talvez sonhar. Ela ouviu o barulho da sirene e quis dizer para o Sr.~Motorista da Ambulância que levasse a mãe dela primeiro. Mas ela só queria dormir.

Seu olhos se abriram e ela viu uma luz forte, algumas pessoas falando e seu corpo todo dolorido e imobilizado.

--- Mamãe? --- ela acha ter dito. \mudanca{Não} tem certeza disso. Os atendentes se olharam.

--- Quem morreu? --- uma voz bem longe perguntou.

Sarah dormiu de novo. E ela sonhou. Sonhou\mudanca{,} sim. Sonhou um sonho onde ela via a si mesma sendo levada de maca para um hospital bem grande e branco onde muitas pessoas choravam, algumas soltavam pulos de alegria,  olhos que se abriam e olhos que se fechavam, havia centenas deles lá.

Ela não sentiu dor alguma. No seu sonho\mudanca{,} viu sua mãe dormindo em uma maca e um homem com os olhos arregalados, tremendo de medo.

--- Morta --- disse alguém.

--- Alguém morreu? --- disse o homem de olhos arregalados. --- Quem morreu? Eu matei alguém\ldots\,Eu matei alguém?

Ninguém lhe respondia. Ela viu um médico e umas pessoas chegarem perto \mudanca{e começarem} a cortar sua roupa e mexer em seu braço. \mudanca{Se} contorceu toda esperando sentir cócegas, mas isso não aconteceu.

--- A vida é um hospital onde quase tudo falta --- lhe disse uma voz amigável ao pé do ouvido. \mudanca{Dessa vez} sentiu cócegas. --- Por isso ninguém se cura, e morrer é que é ter alta.

Tudo estava cheio de amor. Sarah não se sentia mais uma garota. Tão pouco um garoto.  Não se sentia adulta nem criança. Ela se sentia completa.

\mudanca{Acompanhou levarem-na} para um quarto onde todos olhavam assustados para ela, como se não soubessem o que fazer.

--- Nós vamos salvar essa garota! --- o homem de jaleco branco disse aos outros.

Que estranho! Um bando de adultos\ldots\,E adultos sempre sabem o que fazer. Por isso \mudanca{queria} tanto crescer.

--- Prepararem o desfibrilador. Duzentos joules! Carregar!

\mudanca{Sentiu o aparelho tocar sua pele, uma espécie de cócegas por todo seu corpo.}

--- Vê, pequenina? --- lhe disse a voz amigável. --- Estão te chamando! Você deve voltar.

--- Para trás! Trezentos joules! Carregar!

Ela sentiu de novo, mas \mudanca{não} queria voltar. Teria de ser uma garotinha novamente.

--- Eu não quero voltar --- Sarah disse.

--- Mas deve. Sua vida será longa. Viverá mais anos que os seus pais, que é como deve ser.

Ela estremeceu. Perder seus pais? Não. Não. Ela não queria.

--- Trezentos e trinta joules! Carregar!

Ela teve medo. Mas sabia que ali, onde tudo estava cheio de amor, ela estaria bem.

--- O senhor não vai declarar?

A voz amigável a abraçou gentilmente a acalmando.

\mudanca{--- Você terá que ser bem forte. Terá que ajudar seu pai. Terá que ser bem forte --- a voz lhe disse. --- Amar é ver alguém morrer.}

Abriu seus olhos. Sentiu dor por todo o corpo. Havia apenas se deixado levar. Não entedia o que havia acontecido. Mas acordou.

--- Um milagre --- disse alguém. Um homem a olhava com um misto de curiosidade e assombro.

--- Mãe? --- ela perguntou. Antes que houvesse alguma resposta\mudanca{,} voltou a dormir, \mudanca{dessa} vez sem sonho algum.

Era essa lembrança que sempre voltava à sua memória. Que ela já viveu um milhão de vezes mais. Que \mudanca{se} lembrava quando ia ver seu pai dormindo de uniforme com um monte de livros abertos em sua frente e passava a mão em sua cabeça, torcendo para que ele acordasse melhor e com um sorriso no rosto. Onde foi parar o seu pai?

Todo mundo, família e amigos, vinham até ela e diziam que sentiam muito. ``Eu sei o que você está passando''. Ela nada dizia, mas dentro dela havia uma Sarah que gostaria de falar: ``Não, você não sabe! Não sabe nem um pouquinho que seja. Minha mãe morreu! Merda! É lógico que você não sabe o que eu estou sentindo!''.

Seu tio Patrick sempre tentava a animar. E conseguia. Ela sabia que nos olhos deles estava estampada a tristeza, mas ele se esforçava para inventar sempre uma nova forma de fazê-la rir.

Um dia Sarah acordou de noite e foi pegar um copo de água. Seu pai estava conversando com um sujeito chamado Edgar que nos últimos tempos estava indo demais à casa deles. Ela não fazia ideia do porque.

--- Eu já pensei em me matar. Seria fácil. Apertar um gatilho ou comprimidos suficientes. Só uma coisa me manteve afastado desses pensamentos --- disse seu pai ao homem.

--- O que foi? --- quis saber Edgar, fazendo a pergunta que Sarah automaticamente se fizera.

--- Minha filha. Eu lidei muito mal com a morte de minha esposa. Depois eu fiquei meio obcecado em pesquisar sobre o além-morte. Eu realmente queria morrer. Mas eu tenho ela. É meu bem mais precioso na terra. Eu fiquei algum tempo em uma fase profundamente niilista. Não dava valor nem para ela. Mas eu acho que já estou me recuperando. Acho que você nunca se recupera de algo como a morte de um ente querido, mas com o tempo\ldots\,Com o tempo você percebe que VOCÊ está vivo. Então é melhor viver. E viver bem.

--- Ainda bem que você teve sua filha como boia. Pois esse seu \emph{Método Ars Moriendi} será revolucionário. Nem acredito que farei parte disso.

Sarah voltou a dormir. \mudanca{Se} sentiu bem por ter ouvido aquilo de seu pai. Agora ela entendia o que aquela voz havia lhe dito. Havia aquilo sido apenas um sonho? Embora nunca mais tivesse sonhado ou ouvido aquela voz de novo, \mudanca{havia ficado} com uma indescritível sensação de que ela estava \mudanca{consigo} o tempo todo.

--- Queria que sua mãe estivesse aqui --- seu pai lhe disse. \mudanca{Dirigiu} até uma farmácia e cinco minutos mais tarde voltou com a face ruborizada e uma sacola cheia de absorventes.

--- Como eu não tinha ideia\mudanca{, trouxe} com abas, sem abas, com odor, sem odor e até esse aqui que eu não recomendo você a usar --- ele disse segurando um absorvente interno.

Ela riu da cara dele.

--- O que foi? --- ele perguntou.

--- Você está vermelho.

--- Parece que nessa farmácia só trabalham mulheres ---  ele disse\mudanca{,} saindo com carro, divertindo-a com os detalhes sobre todos os olhares estranhos e ele sem jeito, fazendo questão de frisar, ``É para minha filha. É para minha filha''.

--- Você não quer ir para casa? --- Roberto perguntou.

--- Eu prefiro ficar com vocês lá --- Sarah disse. --- Eu gosto de ficar lá.

Sarah não queria deixar de acompanhá-los na Novo Mundo sempre que podia. Tinha dentro de si a sensação de que se um dia deixasse de fazê-lo, perderia alguma coisa. Alguma coisa importante.

Ele a levou para casa onde tomou banho, vestiu uma nova roupa. Preta. Seu armário atual era composto de roupas pretas com exceção de um vestido vermelho que praticamente nunca usou. Foi a única peça de roupa que seu pai comprou para ela depois da morte de sua mãe. Como ele viu a face de desapontamento com a roupa, ele decidiu que Sarah poderia escolher por si mesma ``desde que não seja nada indecente, que mostrasse a barriga ou que sua avô não poderia ver sem ter um enfarte''. Piercing ou tatuagens\mudanca{,} nem pensar\mudanca{, embora} estes últimos não chamassem \mudanca{nem um pouco a atenção de Sarah}.

Quando ela encontrou Falls na Novo Caminho e teve tempo de falar com ela a sós\mudanca{, Falls} explicou por que o absorvente que ela escolhera não fora exatamente a melhor opção. Tudo dependia de vários fatores, como conforto, intensidade do fluxo e outras coisas.

--- E não ache que acabou --- Falls falou para ela.

--- O quê? --- perguntou Sarah irritada com a situação.

--- Com certeza ainda vai descer mais.

--- Deus deve odiar as mulheres. Imagine ter que dar a luz --- Sarah disse. --- Os homens não tem nada disso.

--- Mas quem controla eles? --- Falls disse. --- Quer saber de uma coisa? Nenhum homem cresce mentalmente além dos doze ou quatorze anos. As mulheres ao contrário são mais maturas desde a adolescência.

--- Você aprendeu isso no curso de psicologia? --- Sarah perguntou.

--- Não --- Falls respondeu\mudanca{,} sem jeito, como se isso tirasse toda a credibilidade de sua afirmação, logo acrescentando: --- Observação empírica.

Logo Sarah descobriria que Falls tinha razão. Logo lá estava ela no banheiro reclamando do gato arranhando \mudanca{sua} barriga\mudanca{, tendo} que trocar de absorvente, dessa vez \mudanca{para usar} os certos.

Quando saiu do banheiro\mudanca{,} deu de frente com o sujeito todo atrapalhado que Falls acertou o spray de pimenta \mudanca{ao tentar} parar aquele louco que queria bater em seu pai. \mudanca{Ainda} estava sentido dores. Ficou meio assustado com a forma com que ele a olhou. Parecia que estava vendo um fantasma.

Eles sempre davam carona para Falls, mas não desta vez. \mudanca{Seu} tio disse que ela estava ocupada.

--- Então, você é oficialmente uma mulher, sabe? --- Patrick disse meio sem jeito.

--- Biologicamente.

--- É --- ele confirmou. --- Biologicamente. Você pode ter um filho\ldots\,E\ldots\,E\ldots

--- Tio\ldots? --- Sarah o interrompeu.

--- O quê?

--- Eu tenho educação sexual desde o ano passado.

Ele ficou sem graça.

--- Sua mãe ficava bem alterada nesse período. Brava para valer. Sempre que ela me dizia ``maré vermelha'', eu sabia o que isso significava --- ele disse, fazendo uma pausa constrangedora. --- Eu tinha que comprar chocolate para ela. ``Alimentar a fera'', era como chamávamos. Então\ldots\,Você quer chocolate?

Sarah gostou da ideia. Então foram os dois alimentar a fera.
