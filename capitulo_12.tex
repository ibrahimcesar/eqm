\chapter{Cruzada espiritual}

Eles vieram de lugar nenhum. Cheios de som e fúria, carregavam cartazes e faixas. Não havia mais do que duas dúzias deles\mudanca{,} mas sabiam definitivamente fazer um bom barulho. \mudanca{Estavam} em frente ao número~505 da rua~23, e seus gritos de protesto se dirigiam à Novo Caminho.

Um homem vestido de paletó e gravata cumprimentava a todos os irmãos que chegavam.

--- Paz do senhor, irmão. Fico contente que tenha se juntado a nós nessa cruzada de fé.

Era o Pastor FFF. \mudanca{Estava} com um megafone na mão e olhava para o pequeno rebanho que conseguir\mudanca{a} juntar. Ele deu para cada criança uma ponta de uma faixa onde se lia ``CRUZADA ESPIRITUAL DA IGREJA UNIVERSAL DO REINO DE \mudanca{DEUS}, PASTOR FFF''.

Havia mais algumas placas, escrito coisas como ``MERCADORES DA MORTE'', ``CAPELÕES DO DIABO''. Regina segurava uma escrito ``DEUS ODEIA OS GAYS''. George achou muito estranho.

--- Essa sua placa não está sendo usada fora de contexto?

--- É de uma antiga manifestação --- Regina disse.

--- Mas não é o assunto dessa! --- George observou.

--- A palavra de Deus é eterna\mudanca{,} George.

--- Eu não entendo por que estamos aqui\mudanca{,} afinal as pessoas não têm direito de fazer o que quiserem com as suas vidas?

--- É nosso dever como cristãos alertá-los dos perigos de Satanás, George. Não podemos fingir que o mal não está sendo feito.

--- Mas as pessoas estão pagando. É apenas capitalismo.

--- Que ótima forma de ganhar dinheiro. Por que não vendem droga ou se prostituem também?

--- E o livre-arbítrio? Se as pessoas querem brincar com suas vidinhas miseráveis\mudanca{,} que deixem! --- George disse para ela, impaciente.

Nesse momento, o pastor o abraçou. Ele não tinha ideia de que \mudanca{estava} tão próximo.

--- Deus não joga dados, irmão --- disse o Pastor FFF. para ele, entregando-lhe um cartaz onde se lia ``O TÚNEL DE LUZ É SATÃ''.

``Claro'', pensou George. ``Ele prefere brincar de esconde-esconde''.

O pastor então achou que mais ninguém chegaria. Ele estava acostumado a públicos maiores, mas isto não o abalaria. A Novo Caminho ficava em uma rua não muito movimentada, era um dia de semana e ainda por cima era horário de almoço.

--- Paz do Senhor, Irmãos! --- ele disse \mudanca{ao} megafone\mudanca{,} fazendo um coro de pessoas responderem. --- Estamos reunidos contra essa farsa criada por Satanás, o mestre das mentiras, para arrebatar almas do Senhor. E ele colocou suas garras sobre a mulher de um dos presentes aqui. Ele colocou dúvida em seu coração. Lhe fez promessas. Lhe prometeu o túnel de luz. Mas o túnel de luz, meus irmãos, é a própria manifestação do Príncipe das Trevas, para\mudanca{,} em um momento de fraqueza total, dominar a sua alma. Mas não se enganem! Eu tenho certeza que \mudanca{a serva de nossa igreja percebeu} que tinha sido enganada, e como ela aceitou Jesus em seu coração, como cada um dos presentes aqui, ela não foi enganada. Eu tenho certeza disso, irmãos! Então Deus a levou para junto de si. Eu peço Senhor, que coloque a mão no coração de um homem chamado Doutor Roberto Mouir, que o faça enxergar todo o mal que está causando. Que ele compreenda que o que ele faz é errado. Eu lhe peço\mudanca{, em} nome de Nosso Senhor Cristo\mudanca{,} que tire o encosto que domina esse lugar.

E ele não parava, como uma metralhadora verbal \mudanca{começou} a falar e falar sobre o quanto o sangue de Jesus tinha poder\mudanca{, de} como a Novo Caminho representava o ``Caminho Errado''. Logo, os funcionários perceberam a movimentação lá fora. Como não tinham nenhum paciente naquele momento\mudanca{,} todos foram ver o que estava acontecendo.

--- O que diabos está acontecendo lá fora? --- perguntou Patrick\mudanca{,} preocupado.

--- Era só o que faltava --- respondeu um dos estagiários.

--- Quem são eles? --- Falls perguntou\mudanca{,} assustada.

--- Da Igreja Universal do Reino de Deus. Encontramos nosso nêmesis --- \mudanca{disse um dos} estagiários\mudanca{,} após ler uma faixa carregada por crianças.

--- Como você sabe que eles são da Igreja Universal? --- Patrick perguntou.

--- Pelas faixas, o símbolo. E o sujeito de paletó tem um programa de madrugada naquela rede de televisão que eles controlam.

--- Como você sabe uma coisa dessas? --- Falls questionou ironicamente.

--- Insônia. Falta de programas bons de madrugada --- se desculpou como pôde.

--- Por que eles estão aqui? --- perguntou Roberto\mudanca{,} querendo saber do que se tratava toda aquela balburdia.

--- Pelo que eu entendi o pastor ali diz que \textsc{eqm}s são abominações\mudanca{,} e que eles vieram aqui pelo sangue de Jesus para nos arrependermos de nossos pecados\mudanca{,} ou algo assim. Ele diz que a luz que as pessoas vêem nas \textsc{eqm}s é Lúcifer em pessoa, ou em anjo, sei lá, já que ele é definido como um ``anjo de luz''.

--- Eles estão gravando tudo isso? --- Patrick perguntou ao ver uma equipe de gravação.

--- Claro. É o circo deles. Eu assisti um em que eles fizeram isso em uma clínica de aborto. Ficaram lá por um \emph{bom} tempo --- o estagiário insone disse em tom sério.

--- Como acabou? --- quis saber Roberto que acabara de chegar ali.

--- Eu dormi.

--- Você não disse que tinha insônia? --- outro estagiário perguntou\mudanca{,} em tom de brincadeira.

--- Claro. Mas isso não significa que eu \emph{não} durma. Esses programas onde um sujeito fica declamando versículos são uma espécie de sedativo para mim. Durmo como um anjo.

--- Como um anjo de luz? --- perguntou Falls\mudanca{,} fazendo todos rirem.

--- Só se for um anjo que caiu do céu. E de cara. Se é que me entendem --- observou o outro estagiário.

Quando Jonas viu a pequena multidão em frente à Novo Caminho\mudanca{,} não entendeu direito do que se tratava. Então viu as faixas, cartazes e\ldots\,Regina e George protestando?

--- George? --- ele perguntou\mudanca{,} surpreso.

--- O que você está fazendo aqui? --- George perguntou tão surpreso quanto Jonas.

--- Eu vim ver a Falls --- ele disse\mudanca{,} olhando para o prédio --- Olá\mudanca{,} Regina --- Jonas disse\mudanca{,} sem jeito de cumprimentá-la. Eles não eram muito próximos. Ele nunca deve ter trocado mais do que cinco frases com ela. Ele nem mesmo atendia o departamento dela. George\mudanca{,} de uns tempos para cá\mudanca{,} havia abolido o \emph{joquempô} como método de disputa para ver quem atendia o telefone. Passaram \mudanca{cada um a} ser responsável por departamentos específicos. O de Regina ficou com George. Espere\ldots\,Foi quando tudo fez sentido para Jonas.

--- Vocês estão\ldots\,--- ele não sabia como abordar o assunto. --- Namorando?

--- Não --- respondeu Regina de imediato.

--- Sim --- disse George\mudanca{,} ao mesmo tempo.

Os dois se olharam ambos perplexos.

--- Eu tenho que ir --- Jonas disse apontando para a Novo Caminho. Ele só queria se afastar dos dois. Ele não acreditava que George havia feito segredo sobre isso com ele. Justo Regina? A dama de gelo?

Então Jonas notou que haveria outro problema mais urgente que ele teria \mudanca{que} lidar. Caminhar aqueles metros que separavam a Novo Caminho e os maníacos religiosos \mudanca{seria ter durante esses segundos todos aqueles olhos} sobre ele. George e Jonas tinham uma piada de que eram ninjas sociais. Com sua capacidade de não chamar atenção pela simpatia, aparência ou charme\mudanca{,} eles dominavam a técnica de invisibilidade a tal ponto que uma vez estando em um lugar, ninguém notaria que eles ali estavam. Agora, ele nadaria contra a correnteza disso tudo.

Respirou fundo e caminhou. Seria melhor se ele não olhasse para a multidão. É o que dizem quando se escala uma montanha: ``Não olhe para baixo''. Então, não olhe para a multidão. Mas ele olhou e um calafrio percorreu todo seu corpo. Não era do vento, não era do frio. Então os imaginou todos pelados. O que não ajudou muito. De uma multidão de maníacos religiosos se tornaram uma multidão de estupradores. Ele então caminhou mais rápido, quase correndo.

--- Olá --- disse\mudanca{,} entrando na Novo Caminho. Falls estava com sua bolsa. Ela estava meio apreensiva\mudanca{, mas} chegou perto dele e o abraçou. Trocaram um beijo leve e rápido\mudanca{, o} que\mudanca{,} para Jonas\mudanca{,} já era o máximo. Todos ao redor, Patrick, Roberto e os estagiários\mudanca{,} ficaram um pouco sem jeito com a situação.

--- Você acha seguro sair? --- Falls perguntou.

--- O que eles podem fazer?  --- Jonas retrucou.

--- Nos queimar em uma fogueira --- disse um estagiário\mudanca{,} mordendo em sanduíche.

--- Ou rezar para que Deus mate a todos nós --- disse outro.

--- Eu tenho que buscar minha filha--- Roberto falou com preocupação\mudanca{,} olhando para o relógio.

--- Eu não acho que seja uma boa ideia o senhor sair com todos aqueles malucos lá fora. Afinal eles estão gritando o seu nome --- disse o estagiário que comia um sanduíche.

--- Eu posso buscá-la --- Jonas disse. --- Se o senhor quiser.

--- Eu vou junto --- Falls disse.

--- Você faria isso por mim? --- Roberto disse olhando para Falls.

--- Claro.

--- O que nós devemos fazer é chamar a polícia --- observou Patrick discando o número no telefone.

Jonas e Falls saíram. Novamente Jonas tentou não pensar que eles estavam olhando para ele. E não apenas olhando. Logo buscaram Sarah, que ao contrário do que Jonas imaginava era uma garota falante e até mesmo alegre. Mas continuava toda vestido de preto. Jonas ficou olhando Falls caminhar com ela e pareciam se dar muito bem.

--- Eu vim com um amigo --- ele ouviu Falls dizer.

--- Amigo. Sei. --- provocou Sarah.

Falls abriu a porta para Sarah, depois entrou.

--- Esta é a Sarah --- Falls os apresentou.

--- Olá --- os dois disseram juntos e sem graça, mecanicamente.

Jonas começou a se incomodar com o silêncio que se fez dentro do carro. Ia abrir a boca para dizer qualquer coisa, quando Falls o salvou:

--- Eu contei para Sarah dos loucos na frente da clínica.

--- Você vai encontrar um monte de loucos como psicóloga --- Jonas observou.

--- É --- Falls concordou. --- Mas outro tipo de loucos. Eles são subestimados. Ser louco é como ser eu ou você, só que amplificado.

--- Não parece tão mal --- Sarah disse.

--- Tudo amplificado. A paranóia, manias, obsessões, sofrimento\ldots

--- Eu entendi --- Sarah disse.

--- Então, vamos comer? --- Jonas perguntou para as duas.

--- Claro --- disse Falls. --- Você quer almoçar com a gente?

--- Eu? --- Sarah perguntou para se certificar de que falava com ela.

--- Aham.

--- Tudo bem. Onde você prefere? --- Falls perguntou.

--- Eu? --- Jonas perguntou.

--- Aham.

--- Vocês podem decidir --- Jonas falou. --- Eu só vou querer uma salada mesmo. Como pouca carne. Não como peixe, nem galinha.

--- Você não come peixe, nem galinha? --- Falls perguntou. --- Eu amo galinha.

--- Peixes mijam no mar. E quando eu era menor\mudanca{,} tive dois periquitos australianos.

--- Por que você não come galinha, não entendi?

--- Os periquitos.

--- Não são galinhas.

--- Mas são parentes. Eu não consigo comer os parentes de amigos meus.

--- Você só come comida de coelho? Eu não conseguiria viver sem hambúrguer.

Eles almoçaram juntos. Jonas\mudanca{,} mesmo aproveitando e dando boas risadas\mudanca{,} não parava de olhar para o relógio\mudanca{,} com medo de chegar atrasado. Quando ele virou a esquina da rua~23 notou com satisfação que o pequeno tumulto havia se dispersado. Ele acompanhou as garotas até dentro da Novo Caminho.

--- Com ou sem religião as pessoas bem-intencionadas fazem o bem, e as pessoas mal-intencionadas fazem o mal. Mas que para pessoas de bem façam o mal, basta religião. É por isso que o Estado deve ser laico --- dizia um dos estagiários como se tivesse em um monólogo próprio. Ele percebeu que todos olhavam para ele como se perguntassem, ``Do que diabos esse cara está falando?''. Então ele fechou a boca.

Jonas se aproximou de Falls e disse em seus ouvidos:

--- Me desculpe, mas eu tenho que ir. Você vai ficar bem?

--- Claro --- ela disse, lhe dando outro beijo rápido, leve\mudanca{,} mas que justificava todo o dia de Jonas.

--- Tchau.

--- Tchau.

``Se Deus não existe, tudo é permitido'', disse Ivan Karamazov em um dos livros de Dostoiévski, um daqueles autores de que todo mundo ouve falar, mas ninguém praticamente leu de verdade. Os existencialistas se apropriaram da frase para descrever um mundo onde\mudanca{,} Deus não existindo, tudo seria permitido. Nosso mundo. Mas eles estavam fazendo um erro de cálculo. Um mundo onde tudo é permitido só existe \emph{se} existir um Deus.

 Veja o Gênesis por exemplo, o livro da Bíblia, não a banda com mesmo nome. Deus pôs Abraão à prova, ordenando-lhe que matasse o seu filho primogênito. Se alguém hoje em dia fizesse o mesmo, será que acreditaríamos ser Deus o acusaríamos de ser apenas um velho louco que ouve vozes? Acontece que Abraão seguiu o que a voz lhe disse. Abraão se dispôs a obedecer e, quando já havia pegado a faca para sacrificar seu filho, foi impedido por um anjo, enviado por Deus.

Kierkegaard, um dos primeiros existencialistas e que pode ser definido como um Nietzsche cristão\mudanca{,} por mais paradoxal que pareça, usou esse episódio bíblico e demonstrou que do ponto de vista puramente ético, não se justificaria a prontidão de Abraão. Só a fé justifica a atitude de Abraão. Uma fé que é considerada até mesmo superior à ética.

Somente a fé de Abraão, uma fé cega e inquestionável dos mandamentos divinos é capaz de tornar a ética dispensável. Nega-se a ética por Deus. Então Ivan estava errado. Se Deus existe tudo é permitido. Inclusive matar como ele faria se um anjo não houvesse sido enviado para evitar. E essa interrupção não importa, pois Abraão estava convicto e sem sombra de dúvidas cumpriria a determinação divina. Exatamente como grupos raivosos como aqueles que se juntaram na frente da Novo Caminho ou que treinam homens-bombas para matar dezenas de pessoas. Nosso problema é que nunca nenhum anjo é enviado para evitar.

Quando passou seu crachá e deu entrada após o almoço, Jonas viu que deveria deixar suas reflexões de lado. Entrou na sala olhando para George que estava de alguma evitando o seu olhar.

--- Está fazendo sol hoje --- Jonas apelou para o assunto que se usa quando literalmente não há nada para se conversar.

--- Eu estou namorando Regina em sigilo há algum tempo. Ela pediu que eu guardasse segredo para não fica falada. Ela é muito reservada.

--- Mas você podia ter contado para mim --- Jonas disse.

--- Tshh\ldots\,--- George desdenhou.

--- Eu só fico imaginando como vocês conseguem dar certo sendo tão diferentes.

--- Como assim, diferentes?

--- Ela é religiosa e você é um ateu.

--- Não. Eu não sou ateu. Eu sou niilista, Jonas. Eu não acredito em nada. Ateus são pessoas que não possuem meios invisíveis de suporte.

--- Mas mesmo assim, como dá certo, entre vocês?

--- Ela não sabe --- ele disse meio sem jeito, se ajeitando na cadeira.

--- O quê? --- Jonas perguntou surpreso. --- Você não contou para ela?

--- Era irrelevante. E eu não sei se ela continuaria comigo se eu contasse --- ele falou\mudanca{,} abaixando a cabeça.

--- Eu acho que você deve contar.

--- Há verdades triviais e há grandes verdades. O oposto de uma verdade trivial é claramente falso. O oposto de uma grande verdade é também verdadeiro.

--- O que você quer dizer com isso?

--- Se você acredita não é mentira.

--- Uau --- Jonas disse. --- Você gosta dela --- concluiu\mudanca{,} convicto.

--- Claro --- George disse. --- Eu tive que abaixar os meus padrões de beleza em alguns aspectos, mas\ldots\,Eu gosto dela. Eu até a acho bem inteligente para uma religiosa.

--- Você vai à igreja com ela? --- Jonas lhe perguntou rindo por antecipação.

--- Algumas vezes.

--- E como é?

--- Eu os trato como \emph{se} fossem gente. Ouço o blábláblá e depois comemos algo. Fazemos coisas de namorados. Essas coisas.

--- Só por curiosidade\ldots\,Quando você me disse que era um mito garotas religiosas não faziam sexo, você estava falando isso por experiência própria?

--- Irrelevante --- ele disse\mudanca{,} pegando o telefone que dava seu primeiro toque naquele momento. --- Alô. Já tentou desligar e ligar novamente?

Jonas ficou feliz em ver que George estava se dando bem. Afinal\mudanca{,} era o que eles sempre quiseram. Arrumar namoradas. Ele pensava no quanto isso os iria afastar. Ultimamente George não ia em sua casa todos os dias, nem o chamava para jogar video-game ou assistir filmes. Agora ele entendia o por quê. No começo ficou bravo, enciumado por ele não ter lhe contado. Mas ele o entendia.

 --- Regina e Falls fazem psicologia juntas. Eu dou carona para ela na faculdade --- George falou após desligar o telefone.

 --- Para Falls?

 Ele afirmou com a cabeça.

 --- Eu não achava que ela gostasse de caras como você --- George disse.

 --- Como assim?

 --- Você sabe. Esse olhar perdido, de idiota --- George lhe disse naturalmente. ---  Sabe o que deveríamos fazer? Sair juntos, qualquer dia desses.

 --- Está bem no começo\ldots\,Entre eu e a Falls sabe. Não estamos namorando, nem nada disso.

 Embora Jonas quisesse e muito.

 --- A Regina me disse que ela está atrás do cara certo, sabe? Porque ela já ficou demais com os errados. Ela deve ser \emph{bem} experiente --- George disse\mudanca{,} com um sorriso malévolo.

 O pensamento amedrontou Jonas, \mudanca{que} voltou a trabalhar\mudanca{,} se questionando se ele realmente era o cara certo. Era?
