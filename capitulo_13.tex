\chapter{FFF}

Ele era definitivamente um homem de Deus. Mas nem sempre foi assim. Fernando Ferreira Fonseca, Pastor~FFF\mudanca{,} como é conhecido por todos hoje em dia, já foi um ateu materialista.

Houve um tempo em que ele era médico. Como uma forma de proteção para si mesmo ele se mantinha distante de seus pacientes. No começo foi muito difícil. Ele não suportava ver mães com câncer implorando por mais alguns meses de vida, só o suficiente para verem seus filhos formados, mas que acabam morrendo na mesma noite.

--- De que adianta o eterno criar, se a criação em nada acabar? --- ele sempre desafiava alguém que era crente.

Com o tempo ele foi cada vez mais se afastando deles. Ele não tinha culpa de como as coisas são, afinal, ele só \mudanca{jogava} de acordo com as regras, não criava elas. Por exemplo, ele já teve que recusar atender um paciente por que o mesmo não tinha plano de saúde, mas pôde perfeitamente no dia seguinte fazer a autopsia do sujeito, já que o governo estava lhe pagando para fazer aquilo.

--- Cada aspecto da vida moderna, de seguros de vida, bombeiros, ambulâncias e a medicina moderna são a prova cabal de que não há Deus. Se um dia você se envolver em um acidente de carro e estiver com a perna presa entre as ferragens e tiver que ligar para o resgate ou pedir que Deus o ajude\mudanca{, adivinha} quem te salvaria? Somos todos ateus filhos-da-puta. É o que somos. Uma das coisas que mais me dá raiva é ficar nove horas com o peito de um sujeito aberto, informar a família do desgraçado que ele sobreviveu e a cirurgia foi um sucesso. Quem eles agradecem? A Deus! Deus não teve nada a ver com isso. Nada! Se fossem abolidas todas as leis me mostre quem teria peito de sair nas ruas só dependendo de Deus, me mostre um filho da puta que faria isso!

Mas um dia todas as suas opiniões e crenças foram colocadas de cabeça para baixo. Ele estava na emergência de um hospital quando soube que chegariam três ambulâncias. De uma delas, saiu uma mulher com mais ou menos trinta anos.

--- Morta --- disse um médico que foi examiná-la.

--- Alguém morreu? ---  perguntou desesperadamente um homem que estava sendo tirado de uma maca. Visivelmente alcoolizado, o homem chorava. --- Quem morreu? Eu matei alguém\ldots\,Eu matei alguém?

``O filho-da-puta sai vivo e uma mãe com a vida toda pela frente morreu. \emph{Parabéns}, Universo. Você venceu'', pensou Fernando enquanto caminhava a passos rápidos em direção à terceira ambulância. Era uma garota bem jovem, de cabelos negros. Estava com sangue na franja, havia um corte bem feio. Ele se estremeceu todo. Durante o trajeto da maca para dentro do hospital, \mudanca{lutou} contra um sentimento que ameaçava dominá-lo. Ele estava desesperado. \mudanca{Queria} salvar aquela garota. Era por isso que \mudanca{resolveu} ser médico. Salvar vidas. Mas em algum ponto isso havia se tornado mais uma atitude mecânica, um desafio em que cada vez menos importava quem e apenas o resultado. Salvar uma vida era vencer. Ele só estava interessado em ganhar. Mas dessa vez não. Ele sentiu algo mudar dentro dele.

--- Nós vamos salvar essa garota ---  ele disse para sua equipe que\mudanca{,} desacostumados a ouvirem algo parecido dele, se entreolharam. Na maior parte das vezes ele apenas gritava ordens com todo mundo. ``Agulha~18'', ``Bisturi'', entre outras coisas.

Mais tarde\mudanca{,} Fernando definiria esse dia como aquele em que ele ``voltou a ser humano''.  Em algum ponto do atendimento, as batidas cardíacas da garota foram diminuindo cada vez mais.

--- Nós estamos a perdendo --- disse um membro de sua equipe.

--- Prepararem o desfibrilador. Duzentos joules!

\mudanca{Prepararam} o aparelho passando um gel nos aplicadores. \mudanca{Passaram-no} para ele.

--- Carregar!

O aparelho fez o som característico de quando está carregado e pronto para a descarga.

--- Para trás! --- ele disse para a equipe e descarregou. Nenhuma alteração nos monitores.

--- Trezentos joules. Carregar!

Tentou mais uma vez\mudanca{,} sem sucesso.

--- Trezentos e trinta joules. Carregar!

Então \mudanca{nada} aconteceu. Se fez silêncio no recinto. Uma enfermeira perguntou a ele\mudanca{,} que começava a fazer massagem cardíaca com suas próprias mãos:

--- O senhor não vai declarar?

Declarar a morte. Não. Não. Não! Ele não podia fazer isso. Uma lágrima escapou de seu olho, quando ele começou a aplicar golpes contra o peito da garota. Naquele momento\mudanca{,} enquanto fazia aquilo, já não tinha controle sobre o que ele fazia. Não totalmente. Uma luz rosa inundou sua visão. Depois disso foi como se ele não mais tivesse controle sobre o que fazia. Era como assisti-lo.

Ele se lembrou de quem ele era e de onde estava. Em um piscar de olhos\mudanca{,} tudo voltou à sua memória e \mudanca{não} podia apenas lembrar, como ver. \mudanca{``}Eu vi o mundo como se fosse a época dos tempos apostólicos cristãos, Roma Antiga\mudanca{''}. Durou apenas alguns segundos e logo ele percebeu que agora estava tossindo e os aparelhos registravam a vida pulsante no corpo da menina. Ele se sentiu profundamente tocado. Todos na sala com ele ficaram sem reação.

--- Um milagre --- disse alguém.

Ele ficou eufórico, nunca havia sentido nada sequer parecido com aquilo. Ele sentiu Deus. Certa vez perguntaram a Jung se este acredita em Deus, sua resposta foi: ``Eu sei''. Ele entendeu o que Jung quisera dizer. Entendeu completamente.

Depois de estabilizar a garota\mudanca{,} foi \mudanca{ao} banheiro ficar um pouco só. Enquanto molhava o rosto pensava no que acabara de acontecer. Não era fácil ver tudo o que você acredita escorregar entre o seus dedos. Mesmo sabendo que está profundamente errado, você se agarra àquilo.

Pensou na morte da mãe da garota e que deveria reconfortar o pai dela neste momento tão difícil.

--- Sr.~Roberto? --- ele perguntou \mudanca{a} um homem que parecia muito perturbado em frente a uma sala do hospital. Pela descrição das enfermeiras deveria ser ele.

--- Sim? --- disse o homem com uma voz rouca.

--- Eu sou o médico que atendeu a sua filha quando chegou aqui. Eu acho que o senhor gostaria de saber --- FFF dizia com orgulho por ter tido aquele momento único e precioso de revelação --- que o que aconteceu com sua filha\ldots\,Foi um milagre. Foi uma obra de Deus.

--- Deus? --- perguntou o homem irado. --- Deus não existe. Onde está Deus? Hein? --- ele apontou para a sala --- Onde está Deus agora?

FFF sentiu compaixão por este homem. ``Ele está cego'', ele pensou. ``Tão cego quanto eu já fui. Por que Deus não se revela a todos e se mostra como fez a mim? Será que aquilo que eu passei foi apenas um indicio de esquizofrenia? Eu estou ficando louco e interpretei isso como uma coisa divina?''

--- O mundo --- disse Roberto com fúria --- deve ser entendido como resultado do caos e sorte cega. E se provir de um propósito deliberado\ldots\,Esse só poderia vir de um demônio.

``Terá sido apenas coincidência e não um milagre?'' A garota havia sido salva, disto ele sabia. Mas e aquilo que ele acabara de sentir\mudanca{? ``}O que está acontecendo comigo?''. Ele foi para casa naquele dia extremamente perturbado e mal conseguir dormir\mudanca{,} consumido pela dúvida. Na manhã seguinte chegou a marcar uma consulta com um psicólogo.

O psicólogo o ouviu algum tempo lhe prescreveu alguns remédios e pediu que ele voltasse uma semana depois para uma nova consulta. Isso não resolvia seu problema. A maioria daqueles remédios só ia sedá-lo, botá-lo para dormir. A solução básica da sociedade moderna: se você tem qualquer problema, fuja. Há sempre uma droga que oferece uma saída de emergência.

Fernando parou em frente a uma farmácia. Ele deveria comprar os comprimidos? Sim ou não? Ele ouviu uma música. Era do lugar bem ao lado da farmácia. Igreja Universal do Reino de Deus. O McDonald das Igrejas.

``Por que não?''

Ele caminhou para dentro da igreja, onde já se encontrava em andamento um culto.

--- Se você não paga dízimo --- discursava o pastor no microfone. --- Você está \emph{roubando} de Deus!

Ele se sentou lá no fundo, sem querer ser notado. Sentia-se um peixe fora d'água. No começo achou as palavras do pastor muito duras. Sempre no imperativo.

No final do culto, enquanto todos iam embora, resolveu ficar e tentar conversar com aquele pastor. Não custava nada tentar. Seu psicólogo o tratava como se fosse um doente e FFF sabia, ou ao menos gostaria de pensar\mudanca{,} que não estava.

--- Aconteceu em uma mesa de cirurgia? --- lhe perguntava o pastor deslumbrado.

--- Exatamente.

--- Você gostaria de dar seu testemunho no culto de hoje? --- o pastor lhe perguntou em seu escritório, assim que ele contou tudo o que vinha acontecendo a ele.

--- Testemunho?

--- Isso. Nós pedimos sempre que algum irmão divida com os outros a glória de Deus em sua vida.

--- Eu não sei. Não estou certo de que ainda esteja pronto para me associar a qualquer igreja. O senhor deve me desculpar, mas eu não me sinto bem com a forma com que os fiéis são tratados.

--- Como assim? --- o pastor quis saber se inclinado.

--- Como um rebanho --- FFF falou.

--- Entendo. É que você não está aqui para seguir. Mas para liderar --- o pastor falou para ele. --- Eu também me sentia assim. Nesse nosso mundo as pessoas estão desesperadas pelo conforto de seus anseios e em busca de uma direção. Quando elas encontram uma delas, elas se agarram a isso como se suas vidas dependessem disso. E depende. Só que alguns nasceram para lhes mostrar o caminho. O caminho de Deus. Vivemos tempos medíocres, Fernando. As pessoas já não conseguem mais acreditar que coisas maravilhosas possam estar acontecendo dentro delas mesmas e dos outros. Elas precisam de testemunhos como o do senhor, para acordar.

Então ele subiu para fazer seu testemunho. Suas mãos suaram muito, ele ficou nervoso, os lábios tremendo. Fazia tempo \mudanca{que não se sentia} assim. No começo toda cirurgia que ele participava\mudanca{,} sentia isso dentro de si. Medo de fazer alguma besteira e desagradar aos seus professores. No começo, enquanto ele falava tinha medo de cada frase que soltava, como se fosse dizer algum palavrão ou cometer blasfêmia. Mas no final quando ele se viu cercado pelos sons de palmas e aleluias sendo ditos, e viu o choro na face de uma mulher\mudanca{,} ele se sentiu bem\mudanca{,} ali. Melhor do que jamais se sentira em sua vida. Pela primeira vez desde que saíra da casa dos seus pais\mudanca{,} quando passou no vestibular\mudanca{,} ele se sentia em seu lar.

Logo ele estava fazendo um curso para se tornar pastor e começava a fazer alguns cultos, recebendo elogios de todos ao redor. Em pouco tempo estava sendo designado para cuidar de um novo templo que rapidamente se tornou muito frequentado. Começou a aparecer em um programa que passava nas madrugadas e de dia em um canal evangélico. Em um dos quadros ele chegava a conversar com pessoas pelo telefone que ligavam em busca de conforto e ajuda. Ele estava se tornando muito popular. Chegou até mesmo a repetir um gesto feito muitos anos antes\mudanca{,} e que fez tanto barulho quanto. Ele chutou estátuas de santos em pleno programa.

--- Só há um Deus. \emph{Um} Deus! Quantas pessoas não oram para Nossa Senhora, São Jorge e o diabo a quatro como se eles fossem quem pode fazer algo por eles. Como semi-deuses, pequenas divindades de religiões pagãs. É besteira! Santo não existe. Mas \emph{existe} Jesus Cristo, que pode salvar a sua vida se você aceitá-lo em seu coração. Essa coisa de santo é um politeísmo enrustido. Politeísmo. E está lá nos dez mandamentos. Não terás outro Deus além do Senhor ---  com isso ele chutou uma Nossa Senhora que caiu violentamente no chão, quebrando-lhe a cabeça. --- Por que não aconteceu nada? Por que isso aqui é um ídolo de barro. Quem conhece a Bíblia sabe que o Senhor abomina isso. E estas imagens aqui foram consagradas por padres. Isso significa que eles idolatram isso. Mas não vale de nada.

Nesse momento ele pegou um martelo com o cabo vermelho.

--- Eu irei ser o veículo da ira divina, destruindo esses ídolos de barro a golpe de martelos --- então ele começou a esmagá-los com um martelo. --- \emph{Pregando} a golpes de martelo.

Seus cultos se encheram. Pessoas vindas de todo lugar do país organizavam excursões para ver suas pregações. O Pastor FFF se tornava cada vez mais incisivo. Se ele estava do lado de Deus, seguindo suas leis, por que deveria ficar calado frente aos desrespeitos de Suas leis? Onde há ignorância, ele levaria a Bíblia. Onde haveria pecado, ele levaria a salvação.

Mas nem tudo foram flores. Católicos \mudanca{revoltados manifestavam} em frente de seu templo, recebeu ameaças de morte e até mesmo uma pedrada na cabeça que o deixou desacordado. Mas \mudanca{não} se abalava. Passou a fazer manifestações em frente de clínicas de aborto, contra casas de prostituição e a fazer grandes campanhas beneficentes.

Se manifestou contra aquilo que chamava de ``abominação do homossexualismo''.

--- A lei federal está do nosso lado! Isso é discriminação!

--- Lei federal? --- perguntou FFF. --- Isso é apenas uma instituição humana.

Sabia o nome de quase todos os frequentadores de seu templo e tratava a todos com cordialidade e dando conselhos para os mesmos.

--- Temos que fazer nossa Cruzada Espiritual. Satanás é o mestre das mentiras. E ele está lá fora agora. Talvez esteja engando seu vizinho. Devemos nos preocupar em salvar nossos traseiros e deixar que eles se corrompam\mudanca{?} Não! Faremos de tudo para abrir-lhes os olhos. Não há homem nenhum nesse \mudanca{mundo que} não possa ser salvo. E vocês\mudanca{,} como cristãos, têm essa missão --- ele então abriu sua Bíblia. ---  O caminho do homem de bem é cercado de todos os lados pelas iniquidades do egoísmo e tirania dos homens maus. Abençoados os que, em nome da caridade e da boa vontade, conduzem os fracos pelo vale das sombras, pois ele é o guardião de seu irmão e o que encontra os filhos perdidos. E executarei neles grandes vinganças, com castigos de furor, e saberão que eu sou o Senhor, quando eu tiver exercido a minha vingança sobre eles. Ezequiel, capítulo vinte e cinco, versículo dezessete.

Após estas palavras, encerrou o culto e começou a se dirigir para o seu escritório\mudanca{,} onde já se encontrava uma pequena fila de fiéis para conversar com ele. Os primeiros a entrar foram uma mulher e um homem com os olhos vermelhos.

--- Paz do senhor, Regina --- FFF disse para a mulher.

--- Paz do senhor, pastor. Este é o meu cunhado de quem eu lhe falei.

--- Paz do senhor irmão --- FFF disse para ele. ---  Sentem-se, por favor.

Ele os ouviu. Mas não acreditava. Ele nem mesmo imaginava, como um médico, \mudanca{que} alguém poderia tentar uma coisa dessas.

--- As palavras do senhor, esta noite --- disse o homem com a voz um pouco rouca. --- Me inspiraram. Pareciam ser para mim.

--- Escutem. Isso é grave. É uma abominação --- FFF dizia. --- Vamos pensar em algo para fazer a respeito disso.

Quando FFF chegou em sua casa e foi se deitar, ele não conseguia tirar aquilo da cabeça. Cabia a ele ser o instrumento da justiça divina. Mais uma das cabeças do dragão havia se revelado.

Antes de dormir ele fez uma prece: ``Senhor, dai-me força para mudar o que pode ser mudado. Paciência para aceitar o que não pode ser mudado. E sabedoria para distinguir uma coisa de outra'', ele orou, porque amanhã seria mais um longo dia.
