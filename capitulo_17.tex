\chapter{Mors Ontologica}

Ele a buscou na faculdade. Regina iria pegar carona com ele, mas havia misteriosamente faltado naquele dia.

--- Ela não gosta de faltar. Você a conhece bem?

--- Bem mal --- Jonas disse.

--- Ela é do tipo de garota que diz: ``Eu não gosto de falar nada ruim sobre ninguém, mas\ldots'' Sabe, como é?

--- Sei\mudanca{,} sim. Ah\ldots\,Eu estava pensando se\ldots\,Se você não gostaria de ir na minha casa hoje --- Jonas não sabia como dizer.

--- Por quê? --- ela perguntou.

--- Pixel sentiu saudades --- ele disse.

--- Ele me odeia, Jonas. Admita ---  Falls falou\mudanca{,} colocando o cinto.

--- Ele não te odeia.

--- Não minta para mim --- ela falou\mudanca{,} \mudanca{fuzilando-o} com o olhar.

--- Eu não menti --- sim, ele mentiu.

Ela só tinha ido em sua casa uma vez, quando George e Regina \mudanca{também foram} e passaram quase a noite inteira em disputas altamente competitivas de videogame.

--- Bem vinda à minha Batcaverna --- Jonas falou abrindo os braços. ---  É aqui que eu passo meu tempo debatendo se o copo está meio cheio ou meio vazio ---  Jonas disse.

---  E a que conclusões você chegou? ---  Falls disse\mudanca{,} tirando os sapatos e \mudanca{colocando-os} ao lado da porta.

---  Não precisa\ldots\,--- Jonas começou a dizer\mudanca{,} ao vê-la fazer aquilo.

--- Eu sei. Você já disse isso da última vez. É costume --- ela disse. Ele ficou observando \mudanca{Falls} ficar apenas de meias vermelhas e achou aquilo demais.

--- Minha conclusão é que ele está \mudanca{meio vazio}. Já que\mudanca{,} na maior parte do tempo\mudanca{,} o copo está vazio --- ele \mudanca{comentou,} voltando ao mundo real.

--- O seu copo, pelo menos --- ela emendou.

--- É. O meu copo pelo menos ---  \mudanca{Jonas} concluiu\mudanca{,} reflexivo.

--- Mas o copo foi criado para estar cheio. Não foi? Para que se faria um copo\mudanca{, se não} para colocar coisas nele? Então o copo está meio cheio. Ele foi feito para estar cheio --- ela disse\mudanca{,} abrindo os olhos. --- O que acha?

--- Não sei --- ele disse. --- O que você acha da vida?

--- O que você quer dizer?

--- Eu me faço perguntas, sabe? Todo tipo de perguntas. Por exemplo, qual o sentido da vida\mudanca{?}

--- É estúpido perguntar qual é o sentido da vida --- \mudanca{disse Falls,} achando graça.

--- Não, não é --- Jonas disse\mudanca{,} como se realmente não fosse.

--- Que tipo de resposta você espera? Que algum sujeito de barba branca lhe aponte uma direção e diga com uma voz grossa: ``É por ali, pequeno grilo''\mudanca{?}

--- O propósito da vida, quero dizer.

--- É uma pergunta estúpida, Jonas --- Falls pronunciou em tom sério.

--- Quando você for uma psicóloga\mudanca{,} não vai poder dizer a seus pacientes que eles são estúpidos --- Jonas falou\mudanca{,} ofendido.

--- É. Eu sei. Mas vão \mudanca{estarão}, e a verdade dói.

Ela caminhou até ele, dizendo:

--- Desculpe, não queria te chatear. Eu só consigo pensar em um sentido para a vida\mudanca{,} e ele é \emph{sempre em frente}!

Ela fez cócegas nele\mudanca{,} que retribui. Ficaram trocando cócegas e gargalhadas, até Jonas cair do sofá.

--- Eu ganhei. O copo está meio cheio --- ela disse\mudanca{,} numa pose de vitória.

--- Você é uma otimista, então? --- Jonas perguntou em um tom de desafio.

--- E você, espertinho? Um pessimista ranzinza. Espere! Eu já sei sua resposta, você vai dizer que é um realista.

--- Na verdade\mudanca{,} eu me considero mais um pessimista panglossiano.

--- Você acabou de inventar isso! --- Falls debochou.

--- Não. É sério. Pangloss era um personagem extremamente otimista de \emph{Cândido}\mudanca{,} do Voltaire. E ele vive dizendo que ``este é o melhor dos mundos possíveis''. Uma paródia de Leibniz, mas você, pela sua cara, parece não ter ideia de quem seja\ldots\,Bem, o pessimismo panglossiano diz que dado que este é o melhor dos mundos possíveis, melhor não dá para ficar. O melhor dos mundos possíveis não implica necessariamente que ele é bom ou aceitável. Eu, por exemplo, se fosse comprar um Universo e recebesse este, \mudanca{devolveria}.

--- Você acredita em Deus, Jonas? --- Falls pergunto\mudanca{, olhando-o}.

--- É complicado. Um dos meus tios, uma vez\mudanca{,} serviu o exército\ldots\,--- Jonas sempre encontrava dificuldade em responder essa pergunta.

--- Você têm quantos tios?

--- Uns seis. Por quê?

--- Família grande, hã?

--- É. Acho que é. A sua é pequena?

--- Meus pais são ambos filhos únicos. Eu tenho uma irmã, mas a gente não se fala muito. Mas continue, seu tio que serviu o exército\ldots

--- Então, ele contou que uma vez estava para chegar um novo capitão no batalhão \mudanca{em} que ele servia. Antes mesmo que o tal capitão chegasse, circulavam \mudanca{boatos} de que era um sujeito rígido, fechado e autoritário. Quando\mudanca{,} enfim\mudanca{,} chegou o capitão e encontrou com a tropa, sabe qual foi a primeira coisa que ele fez? \emph{Contou uma piada}. Mas ninguém riu\mudanca{,} pois estavam com medo, não queriam rir na frente dele. Achavam que o sujeito era uma coisa e na verdade era outra\mudanca{,} totalmente diferente. É assim que eu vejo Deus. Como uma espécie de comediante \mudanca{cuja} plateia tem medo de rir\ldots\,Não diz muita coisa, mas é como me sinto\ldots\,E você?

--- Eu só acredito em Deus como responsável pelas coisas ruins na vida ---  ela disse. --- Eu não consigo acreditar em um Deus que se importa com que eu como no café da manhã, sabe? Eu não ligo para religião.

Ela ficou um tempo em silêncio\mudanca{,} antes de voltar a se empolgar dizendo:

--- Estou pensando em lagar meu emprego e começar um estágio, com crianças. Já fui ver um lugar para estagiar.

--- Sério, com crianças?

--- Você não gosta delas? De crianças?

--- Está mais para elas não gostam de mim --- ele respondeu\mudanca{,} meio que fugindo do assunto.--- Quer ver um filme?

--- Claro. Qual? --- Falls perguntou.

--- É o meu filme predileto. Talvez você já tenha visto, então eu coloco outro. É \emph{Brilho Eterno de uma Mente sem Lembranças}.

--- Não, nunca vi. É antigo?

--- É do tipo que não envelhece.

Então eles assistiram o filme, absortos na trama, de vez em quando Falls adicionava alguns comentários. Quando o filme acabou e os créditos rolaram ao som da canção, Falls se virou para Jonas e\mudanca{,} olhando dentro de seus olhos\mudanca{,} lhe perguntou:

--- Quem você apagaria? Se eu te deixasse, você me apagaria?

---  Eu não sei. Mas\ldots\,A mensagem do filme não é que não adianta apagar as memórias já isso não é lidar com os problemas e sim fugir deles? Todos os personagens que passaram pelo processo em algum ponto acabam recebendo tudo de volta. Nossa memória é o que nós somos. É meio cliché falar isso, e é meio cliché anunciar que vai dizer um cliché, mas nós aprendemos com nossos erros.

---  Quais são seus erros?

---  Se eu começasse a dizer agora provavelmente nos atrasaríamos para o trabalho\mudanca{,} então acho melhor nem começar ---  Jonas disse, subitamente \mudanca{lembrando-se} de algo que o deixava acordado algumas noites. Algo que ele pensava sempre. Do dia em que ele se deu conta de que nunca havia dito ``eu te amo'' a ninguém, já se fazia alguns anos. No dia em que seu pai morreu. ``Sempre tão perto\ldots\,Tão distante'', ele pensava de todas as garotas que faziam sua cabeça. Agora ele estava ali, olhando para uma garota linda\mudanca{,} que podia ter qualquer outro cara em um minuto. Era só dizer. Três simples palavras. Parecia ser fácil. Lhe faltava a voz. Ele gaguejava. Ele precisava dela tão mais perto.

--- O quê? --- Falls lhe perguntou, notando o embaraço dele, ecoando na memória de Jonas a primeira vez que ela a viu. --- Não vai me dizer que horas são? --- ela disse, sorrindo. ``Covinhas'', ele pensou.

--- Euteamo --- ele disse, assim, de uma vez. Se sentiu um bobo, respirou fundo. --- Eu te amo.

Ela abriu um sorriso. Então eles se beijaram calmamente, sem pressa. Ele sentiu como se suas moléculas estivessem se mesclando com as delas, eles estavam tão unidos \mudanca{quanto} as moléculas de carbono no dióxido que eles formam.

--- Eu te amo --- ela disse. --- De verdade. Eu já me enganei algumas vezes. Mas agora eu estou certa. Onde você esteve esses anos todos? --- ela perguntou com uma voz doce em seu ouvido.

--- Esperando\ldots\,--- ele disse para ela. --- Esperando.

Eles ficaram abraçados.

--- Eu posso dormir aqui, hoje? --- Falls perguntou.

--- Claro --- Jonas falou\mudanca{,} imaginando o que ``dormindo'' queria dizer

--- Mas você não vai poder fazer nada, seu tarado. Nem fique animado --- ela disse\mudanca{,} respondendo às suas dúvidas secretas. --- Se você tivesse me dito essas coisas dois atrás eu não estaria\ldots\,Você sabe\ldots

Ele vagamente sabia do que ela falava\mudanca{, mas} não queria estragar aquele momento. Ele fechou seus olhos e a abraçou com força. Então, tentou repetir as palavras ditas no filme, palavras que ele havia decorado fazendo secretamente uma prece para que o Universo lhe concedesse a chance de um dia ter alguém.

--- Eu poderia morrer agora, Falls. Estou tão feliz. Nunca me senti assim antes. Estou exatamente onde eu queria estar --- ele disse\mudanca{,} passando suavemente a mão pelos cabelos dela. Tão bom morrer de amor e continuar vivendo.

Então eles foram para cama onde dormiram bem juntinhos, de conchinha. Jonas adormeceu logo com um sorriso no rosto pensando que não poderia ficar melhor que isso. Dormiu profundamente.

Quando ele acordou \mudanca{pela}  manhã, passou a mão pela cama e abriu os olhos preocupado. Ela não estava ali. Ele foi para a cozinha, onde havia água fervendo no fogo e Pixel estava comendo sua comida.

Quando Falls se levantou, procurou não fazer barulho para acordá-lo. Pixel a observava com olhos curiosos. ``Então é uma pessoa de gatos?'', ela pensou\mudanca{,} olhando para Jonas dormindo. Abriu o armário dele e pegou uma camiseta. Vestiu-a e foi ao banheiro. Quando deu descarga percebeu que ele havia levantado. ``Eu vou assustá-lo'', ela pensou, sempre disposta a pregar peças.

--- Boo! --- \mudanca{Falls apareceu,} saltando na frente dele, detrás do sofá. Ele deu um pulo para trás.

--- Jesus --- ele disse, colocando a mão no peito.--- Que susto\ldots

Ele nem mesmo conseguia ficar bravo vendo ela sorrir com tanta propriedade e gosto. Até mesmo ele acabou caindo no riso. Tomaram o café juntos, e Falls quis ajudá-lo a escolher a roupa que usaria no trabalho.

--- Quem compra suas roupas? --- ela disse\mudanca{,} olhando o armário com um olhar desconfiado.

--- Minha mãe\ldots\,Como você sabe que eu não\ldots

--- Está na cara, Jonas. E sua família? Você ainda não me disse nada sobre sua família --- ela falava\mudanca{,} mexendo em suas camisas.

--- Eu tenho um irmão que é casado e mora em outra cidade. E\ldots\,Bem, meu pai morreu e minha mãe me trata como se eu ainda tivesse cinco anos de idade.

--- Deu para notar pelas roupas --- ela disse\mudanca{,} sorrindo.

Ele guardou aquele sorriso com ele durante todo o dia. Após o almoço, no trabalho, Jonas\mudanca{,} após ver os novos jogos que George \mudanca{havia} comprado, resolveu contar a ele a mais incrível noite de sua vida.

--- Eu levei ela em casa ontem e\ldots

--- Dormiu com ela? --- George perguntou ansioso, com um sorriso de gato de Cheshire, que ia de ponta a ponta de seu rosto.

--- Eu ia dizer assistimos um filme, mas\ldots\,Dormimos juntos, sim. Pode-se dizer.

--- Como assim, \emph{pode-se} dizer?

--- Bem, nós caímos no sono no mesmo cômodo e então, tecnicamente, nós dormimos juntos.

--- Tshh --- George desdenhou. --- Você é patético, Jonas. Patético.

Quando chegou em casa, fez um jantar rápido e partiu para folhear o \emph{Mors Ontologica} que Edgar havia \mudanca{emprestado}. Não era um livro muito grande. Sentou-se no sofá, ligou a televisão, apertou o botão \mudanca{mute} e começou a ler. Ele tinha esse costume de precisar ligar a televisão estando na sala, mesmo não querendo olhar para ela. Jonas tinha planejado ler um, dois, quem sabe três capítulos naquela noite e depois ir dormir mais cedo\mudanca{,} porém, quando deu por si\mudanca{,} estava fechando o livro às cinco da manhã\mudanca{,} com os olhos pesados de sono. Não conseguiu dormir nem mesmo duas horas já que o despertador, lá do quarto, começou a tocar. Quando se levantou sentiu uma tremenda dor nas costas por ter dormido sentado\mudanca{,} e tomou a decisão de somente ler na cama a patir de agora. Desligou a televisão e foi colocar a comida de Pixel em seu prato.

As ideias contidas no livro ainda estavam frescas em sua memória. Algumas partes ele não entendeu completamente\mudanca{,} mas tanto quanto pode parecia ser uma teoria bem complexa e que não se restringia apenas a experiências de quase-morte.

Ele defende uma teoria de que \mudanca{existem} oito circuitos da consciência. Foi pela primeira vez formulada por Timoty Leary e defendida por Robert Anton Wilson, \mudanca{que é citado} no livro\mudanca{,} a fim de demonstrar tal teoria\mudanca{, que presumia} que nosso sistema nervoso consiste de oito circuitos \mudanca{em potenciais, ou} ``engrenagens'', ou \mudanca{ainda} minicérebros. Quatro desses são os normalmente usados e responsabilizam-se pela nossa sobrevivência terrestre. Quatro são extraterrestres, silenciosos e inativos.

O primeiro circuito, \mudanca{chamado} de biossobrevivência, um cérebro invertebrado que foi o primeiro a evoluir, a cerca de dois a três milhões de anos atrás, é o primeiro ativado quando uma criança nasce. Ele aborda e categoriza as coisas que \mudanca{a criança toma conhecimento} e aceita e aquelas das quais foge e ataca. É a consciência \mudanca{per se}, o sentido de estar aqui, agora, orientado para a sobrevivência do corpo.

O segundo circuito é o emocional e formou-se quando os vertebrados apareceram e começaram a competir por território, talvez quinhentos milhões de anos atrás. Ele é ativado quando se passa da forma de rastejar para andar. Como todos pais sabem, o bebê não é um ser passivo, mas um mamífero político\mudanca{,} cheio de exigências territoriais físicas e psíquicas\mudanca{,} que interferem rapidamente nos assuntos da família e na tomada de decisões. \emph{Grosso modo}, pode-se dizer que é o ego.

O terceiro é o circuito da destreza-simbolismo\mudanca{,} e foi formado quando as espécies hominídeas começaram a se diferenciar dos outros primatas\mudanca{, sendo} ativado mais tarde\mudanca{,} quando a criança começa a manipular objetos e usar a linguagem. O terceiro se enquadra daquilo que geralmente denominamos mente, ou seja, a capacidade de receber, integrar e transmitir sinais.

O quarto circuito é o sociossexual e foi formado há uns trinta mil anos\mudanca{,} quando nossos bandos hominídeos evoluíram para sociedades e definiram os papéis sexuais específicos para os seus membros. É ativado na puberdade.

Esses quatro primeiro são normalmente todas as redes já ativadas por qualquer pessoa que você cruze na rua. Os outros quatros estão em evolução.

O quinto, \mudanca{chamado} neurossomático\mudanca{,} quando \mudanca{ativado}, explode multidimensionalmente as configurações euclidianas básicas. A pessoa se liberta das prisões sensoriais. É por esse motivo que dizemos que alguns usuários de certas drogas estão ``viajando''. A transcendência das orientações espaciais, lineares, aristotélicas, newtonianas. O estado é descrito como ``flutuando'', ou\mudanca{,} para utilizar de uma metáfora zen, ``um pé acima do chão''.

O circuito neuroelétrico \mudanca{manifesta-se} quando o sistema nervoso se torna perceptivo de si mesmo. O circuito neurogenético se refere a visões do passado ou do futuro. Ao inconsciente coletivo de Jung, a relatos de toda a vida por aqueles que passaram por experiências de quase-morte.

O neuroatômico é de longe o mais ousado circuito que a inteligência jamais ousou aventurar-se: a consciência precede a unidade biológica. Experiências fora do corpo, projeção astral, contato com alienígenas.

\mudanca{O livro,} então\mudanca{,} explora as experiências de quase-morte \mudanca{do} ponto de vista de que\mudanca{,} ao \mudanca{ficarmos} perto da morte, nós acionamos todos os oito circuitos e atingimos o potencial da humanidade. Mas \mudanca{não} aponta muitas conclusões. São experiências de \emph{quase}-morte. Não se sabe se\mudanca{,} ao morrer \emph{realmente}\mudanca{,} essa experiência exista.

Mesmo hoje, com o uso de certas drogas e rituais, ou experiências próximas da morte\mudanca{,} podemos transcender os quatro circuitos ordinários da consciência. Isso explica\mudanca{,} em parte, o anseio de milhares de pessoas em partir para o uso de drogas. Dizemos que eles buscam uma fuga para a fantasia. Mas com que certeza podemos dizer \mudanca{que} não é apenas uma realidade mais elevada? Jonas não sabia sua opinião a respeito\mudanca{;} na verdade\mudanca{,} ele queria ter umas boas horas de sono antes de pensar qualquer coisa, mas não teria tempo para isso, afinal tinha que trabalhar.

Quando Jonas entrou bêbado de sono na sala, \mudanca{cumprimentou} George com uma voz pesada.

--- Passou a noite em claro? --- George perguntou.

--- Aham --- Jonas disse\mudanca{,} colocando as mãos na cabeça, com uma tremenda dor de cabeça. --- Estou acabado, cara.

--- Dormiu com ela? --- George perguntou\mudanca{,} sorrindo.

--- Não. Não dormi --- Jonas disse.

--- Tshh\ldots\,Você é patético, Jonas. Patético.

--- Sexo não é tudo George --- Jonas disse de olhos fechados, apreciando a sua dor.

--- Impérios vêm e vão, ids explodem, grandes sinfonias são compostas, ataques terroristas são feitos, alguém faz uma prece a algum dos milhares de Deuses, mas\mudanca{,} por detrás disso tudo\mudanca{,} está um único instinto que requer satisfação.

--- Qual? --- Jonas perguntou\mudanca{,} ironicamente.

--- Tshh\ldots\,--- George voltou a ler seu jornal e tomar café de uma xícara. Jonas apenas colocou a cabeça na mesa por um tempo, tentando não se render ao sono que \mudanca{o ameaçava} fazer seu prisioneiro. Jonas olhou para o jornal de George, que estampava em letras garrafais uma grande tragédia, com um número de mortos de mais de seis dígitos. A chamada ainda prometia fotos dos mortos no acidente.

--- Isso é mesmo ultrajante. Uma tragédia --- George disse\mudanca{,} em tom sério e preocupado. --- Digo, pelo amor de Deus\mudanca{,} como eles podem deixar isso continuar acontecendo! Isso é terrível. Eles precisam fazer alguma coisa quanto a isso. Eles precisam!

--- O que aconteceu, George? --- Jonas perguntou, levantando a cabeça, meio preocupado.

--- Parece que o Corinthians vai ser novamente rebaixado! --- George disse\mudanca{,} desolado\mudanca{,} mostrando o caderno de esportes\mudanca{,} que ficava no meio do jornal.

--- Um niilista que precisa apelar para Deus? Então a situação está crítica mesmo.

--- Eu não tenho culpa da minha educação, Jonas. Cresci tendo que ir \mudanca{ao} maldito catequismo. Tive sorte por nenhum padre ter tocado em meu traseiro.

Jonas apenas riu.

--- Mas não posso dizer o mesmo dos coroinhas.
