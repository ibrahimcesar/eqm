\chapter{Eu-Tu}

Mais uma vez os Renascidos estavam se encontrando. Edgar e Frederico não eram apenas pontuais, \mudanca{mas também} chegavam um bom tempo antes do horário marcado, o que fazia que eles se tornassem mais próximos um do outro, \mudanca{mas não} necessariamente \mudanca{amigos}.

--- Eu fico me perguntando\mudanca{,} o que te faz chegar tão mais cedo? --- Frederico perguntou a Edgar\mudanca{,} estendendo-lhe a mão.

--- Eu não gosto de ficar no meu apartamento. Há pensamentos demais lá --- Edgar se queixou.

--- Sabe o que eu faço? Ligo o som no máximo. Quando caminho, escuto música. Isso te liberta do fardo que é pensar.

--- Seus vizinhos devem adorar --- Edgar observou.

--- Eu já pago um condomínio bem alto para não ter o direito de ouvir a música no volume que eu quiser. E o rapaz, Jonas\mudanca{,} virá hoje?

--- Ele me disse que sim --- Edgar falou\mudanca{,} abrindo sua pasta.

--- O que você acha que aconteceu com ele? Quero dizer, cada um tem seus motivos para buscar algo assim.

--- Ele perdeu sua fé. Foi isso que aconteceu.

--- Olhe ao redor. Mil e um motivos para acreditar? Não, muito pelo contrário.

--- Talvez você só esteja olhando errado, só isso.

--- Você me faz rir, Edgar.

--- O que você queria, por acaso? Algum tipo de cruz gigante no céu ou criminosos sendo instantemente fulminados por raios divinos?

--- Bem, tenho que concordar --- Frederico disse após refletir. --- Isso já seria um bom começo. Se Deus existe que ele \emph{mate todas as pessoas que estão lendo nesse exato momento} --- Frederico falou\mudanca{,} olhando para um sujeito em uma mesa próxima com o jornal aberto. --- Viu? Nada.

Então Jonas chegou e os cumprimentou. E tirou de uma bolsa o livro \emph{Mors Ontologica}.

--- Então, o que achou? --- perguntou Edgar.

--- Eu achei realmente muito interessante --- Jonas falou\mudanca{,} com medo de dizer algo estúpido. Na verdade\mudanca{,} passou todo o caminho tentando decorar algumas coisas para dizer. ---  Naquela parte em que se discute anjos, demônios, alienígenas, as criaturas de luz vistas nas \textsc{eqm}s\ldots\,Você acredita mesmo que possa ser nós mesmos no futuro?

--- É possível, mas quem sabe? Eu tenho certeza absoluta de que é um erro ter certezas absolutas. Quanto mais leio, mais me dou conta de que parece que nós apenas damos voltas. Mudamos o vocabulário, claro. Refinamos nossas ferramentas. Por Deus, colocamos o homem no espaço! Mas parece que nosso conhecimento sobre a realidade continua o mesmo de textos milenares. O Big Bang, por exemplo. É igual ao grande ovo cósmico de diversas civilizações, como a egípcia. O conceito de que o universo se expande, depois se contrai está no ``respiro do cosmo'' dos budistas. Eles até mesmo calcularam com precisão o período e quase se encaixa ao que nos diz a ciência moderna. Agora mesmo\mudanca{,} a ciência está chegando a não existência do ``eu'', como há muito tempo pregavam os hindus. Séculos de inovação e o grande conselho está nas palavras finais do \emph{Bardo Thödol}: ``Entra no jogo existência, com boa graça, voluntariamente e livre, permanece tranquilo''.

--- Eu acho\ldots\,Olá a todos --- disse Arthur quando chegou, entrando na conversa\mudanca{,} vindo de lugar nenhum. --- Eu acho que exista uma espécie de lugar para onde vamos. É o mesmo lugar dos registros Akáshikos, ou onde vamos quando sonhamos. Uma vez eu li que os usuários de \textsc{dmt} descrevem alucinações ricas em detalhes e muito parecidas, como se eles fossem para uma outra dimensão, decodificada na droga. Uma realidade compartilhada. Um cientista, Dr. Karl Jansen, conseguiu induzir os mesmos efeitos de uma \textsc{eqm} através de uma droga chamada quetamina. Pensamentos coletivos criando a realidade. Mas nesses estados mentais, em \textsc{eqm}s, sonhando, tendo convulsões, estando chapado, você se liberta da prisão dos sentidos e tem acesso a outro nível da realidade. Oh\mudanca{,} malditas fronteiras da linguagem\mudanca{,} que não me permitem expressar\ldots

--- Acho que o que está querendo dizer --- Edgar interveio. --- É que a realidade é bem maior do que nossos cinco sentidos e mesmo nosso cérebro que interpreta esses cinco sentidos é capaz de perceber. Quando passamos por certos eventos então acessaríamos ela. Mas por que há divergências sérias, como um ir para algo parecido com o Céu e o outro para algo parecido com o Inferno\mudanca{?} Ou alguns verem Hare Krishna e outros verem John Lennon?

--- Isso se explicaria por toda a bagagem cultural, biográfica e preconceituosa do indivíduo. Por isso que se busca com tanto desejo a morte do ego --- Arthur retrucou.

--- Morte do ego. Eu nunca entendi essa --- Jonas os interrompeu.

--- É a anulação daquilo que chamamos \mudanca{``eu''}. O que é o ``eu''? Temos um centro que toma as decisões em nossa vida ou somos apenas dominados por ideias, desejos e sensações, uma máquina de memes? É extremamente complexo. A ciência avançou muito depois que eu escrevi o \emph{Mors Ontologica}. Estou pensando em atualizá-lo, mas o trabalho na Novo Caminho não me dá tempo. Mas mesmo as \textsc{eqm}s estão impregnadas de ego\mudanca{,} já que seu conteúdo se explica como parte do conhecimento histórico, crenças, elementos de seu próprio ser reprimidos, fantasias, desejos ocultos e até mesmo da última vivência diária.

--- Sabe o que eu suspeito, Jonas? --- Frederico lhe perguntou.

--- O quê? --- Jonas perguntou.

--- É o tipo de suspeita hipotética que se tem de vez em quando, uma ideia maluca --- Frederico disse. --- Eu acho que talvez o mundo em que vivemos seja o Inferno. Eu avisei sobre ser maluca, não me olhe assim\ldots\,Pense bem, estamos aqui para pagar pelos pecados de uma outra existência. Isso explicaria muita coisa. Por que há \mudanca{sofrimento em demasia} e \mudanca{é} tão difícil ser feliz, por exemplo\mudanca{?} Se um dia eu descobrisse que o gênio maligno de Descartes está por trás da nossa vida, isso explicaria muita coisa. Explicaria mesmo.

Mais uma vez Jonas adorou ter participado da reunião. Ele adorava respirar aquele ar impregnado de sabedoria que aqueles sujeitos pareciam emanar. \mudanca{Porém,} cada vez mais, aquilo era ofuscado e parecia não ter relevância alguma. Ele olhava para Falls preparando biscoitos e pensava no quanto ela acalmava seus demônios questionadores.

--- Espero que você goste, eu aprendi com a minha avó --- ela disse.

Ele pegou uma pequena moldura que tinha a foto de uma criança linda. \mudanca{Ela} sorria para a foto. ``Covinhas'', \mudanca{pensou}.

--- É minha filha --- Falls disse\mudanca{,} ao vê-lo com a moldura. Jonas ficou confuso por um momento. --- É a minha sobrinha, bobo --- ela disse rindo.--- Mas eu adoraria se tivesse uma só para mim\mudanca{,} sabe. Uma vez eu pensei em adotar. Eu sei me virar sozinha, então\ldots

--- Então você sabe se virar sozinha? --- Jonas perguntou, de brincadeira.

--- Eu sou uma sobrevivente --- ela falou\mudanca{,} confiante. ---  Eu sei me virar muito bem sozinha.

--- Uau --- Jonas disse. --- Uma sobrevivente?

--- É sério --- ela falou, se sentando. --- Eu quase me enforquei com o cordão umbilical antes de nascer. Tiveram que me tirar de dentro um pouco mais cedo.

--- Isso soa mais como uma suicida do que como uma sobrevivente --- Jonas disse\mudanca{,} começando a rir\mudanca{,} embora não sem sentir um pouco de tristeza pelo peso que a palavra ``suicídio'' tinha em sua vida.

--- Seu chato --- ela disse, se fazendo de magoada.

--- E você, sua linda --- ele disse, fazendo-a sorrir. \mudanca{``Covinhas''}.

--- Vai ter troco --- ela disse.

--- O quê? --- Jonas perguntou.

--- Eu não sou uma vaca vingativa, se é isso que está pensando. Eu apenas acredito em reciprocidade, sabe? Você coça as minhas costas, eu coço as suas. Você me dá um chute, eu te dou outro. Carma.

--- Então, você vai me dar um chute? --- ele falou\mudanca{,} fingindo preocupação.

--- Se você me der um\ldots\,Claro --- Falls olhou para o forno e viu que seus biscoitos estavam prontos. --- Estão prontos!

Ela vestiu sua luva térmica e tirou os biscoitos. Depois começou a passar os biscoitos para um pequeno pote. Jonas estava lá olhando para ela, cada um de seus movimentos, tendo um prazer inenarrável em ser um exclusivo expectador de tal cena.

--- Eu não me sinto muito confortável com você aí, me olhando --- ela disse\mudanca{,} olhando para ele.

--- O que você acha da Novo Caminho? --- Jonas perguntou.

--- É só o que eu faço para ganhar meu próprio dinheiro --- ela respondeu.

--- Mas, eu digo\mudanca{, dessa} coisa das pessoas arriscarem suas vidas só para conseguir um propósito na existência.

--- Todos morrem um dia. É triste saber que elas precisam realmente apostar tudo para se darem conta desse fato tão óbvio --- Falls disse\mudanca{,} ainda tirando os biscoitos.

Ela pegou o pote de biscoitos.

--- Vamos assistir alguma coisa --- ela disse\mudanca{,} guiando\mudanca{-o} até o sofá\mudanca{,} onde eles se sentaram. Ela colocou os biscoitos no meio dos dois. --- Vamos, experimenta.

Jonas pegou um dos biscoitos e o comeu.

--- Bom. Não tão bom quanto seu beijo, mas mesmo assim, gostoso.

Falls riu.

--- Isso é alguma espécie de convite?

Eles se beijaram.

--- Hoje eu tive uma aula muito interessante e\ldots\,--- Falls não conseguiu concluir\mudanca{,} pois Jonas a beijou novamente. --- Eu não consigo falar enquanto você coloca a língua na minha garganta.

--- Me desculpe! --- Jonas falou em um tom jocoso, levantando os braços. --- Não vou por mais minha mãos em você.

--- Você vai poder fazer isso pelas próximas horas, eu só queria te dizer uma coisa. É importante.

--- Tudo bem --- Ele disse\mudanca{,} se aproximando dela e esperando que \mudanca{começasse} a falar.

--- Você não está bravo?

--- Pelo quê?

--- Você veio para cima de mim e\ldots

--- Não\mudanca{,} tudo bem. Eu posso esperar. E eu estou com você, do que posso reclamar?

--- Voltando ao que eu queria dizer, esse pensador, Martin Buber\mudanca{,} falava de dois modos de expressão entre as pessoas. Ele falava da comunicação, entende? A comunicação se expressa de duas formas. EU-TU e EU-ISSO. O EU-ISSO é usado em nossas relações com o mundo das coisas. Essa é a minha casa. EU-ISSO, entende? O EU-TU são nossas relações com seres humanos. Eu quero passar o resto de minha vida com você. EU-TU.

--- Você quer mesmo?

--- O quê?

--- Passar o resto da sua vida comigo? Os cabelos caírem e tudo mais?

--- Foi só um exemplo, mas eu acho que sim --- ela respondeu\mudanca{,} meio constrangida, mas bem consigo mesma por ter dito aquilo. --- Mas o que se aprende? Que as relações entre as pessoas podem ser objetais. Sabe? As pessoas tratam as outras como objetos. EU-ISSO. E isso acontece o tempo todo. Mas sem o ISSO, o homem não pode viver. Mas só com isso não há relação verdadeiramente humana possível. Geralmente as relações começam no EU-ISSO e terminam no EU-TU. E se não fazem essa transição a tendência é que essa relação se deteriore. Sabe? Posso te fazer uma pergunta?

--- Duas, porque uma você já gastou.

--- Espertinho --- ela disse\mudanca{,} olhando para ele e passando a mão em seu cabelo. --- Por que você me ama?

--- Eu\ldots\,--- Jonas não sabia dizer. --- Eu não sei. Eu só sei que eu te amo.

Falls abriu um sorriso.

--- Obrigado --- ela disse.

--- E o que eu fiz?

--- Meu professor disse que se você conhecer alguém que diz amar outra pessoa e perguntar por que essa pessoa ama a outra e ela responder, a verdade é que ela não ama. Explicar o amor por uma pessoa seria EU-ISSO e não existe relação verdadeiramente humana no EU-ISSO.

--- Eu nem chego acreditar que você é minha --- Jonas disse.

--- EU-ISSO --- ela falou em um tom de reprovação.

--- O quê? Desculpe. Eu não queria. Desculpe, por favor me desculpe --- ele disse logo, a abraçando.

--- EU-TU\ldots\,--- ela pronunciou no pé do ouvido de Jonas, iniciando um arrepio que percorreu todo \mudanca{seu} corpo.

Eles sorriram e voltaram a se beijar até que Falls parou\mudanca{,} antes que o pote de biscoitos entrasse em sua costela. Os biscoitos haviam se espalhado pelo sofá, Jonas até mesmo havia esmigalhado alguns deles sentando em cima.

--- Nós desperdiçando esses biscoitos e crianças pelo mundo se matariam por isso ---  Jonas disse.

--- É a ironia do mundo. Enquanto metade morre de fome\mudanca{,} a outra metade tenta emagrecer --- Falls falou enquanto\mudanca{,} os recolhia.

Jonas estalou os dedos.

--- Assim que eu estralo meus dedos, uma pessoa acaba de morrer de fome. --- ele estalou novamente. --- Outra acaba de morrer. Outra agora --- ele disse enquanto estralava o dedo. --- E mais outra --- disse, também estralando.

--- Então pare de fazer isso! --- Falls disse o interrompendo um estalo que ele iria fazer, fazendo os dois caírem na gargalhada.

Impérios vêm e vão, ids explodem, grandes sinfonias são compostas, ataques terroristas são feitos, alguém faz uma prece a algum dos milhares de Deuses, crianças morrem de fome, mas por detrás disso tudo está um único instinto que requer satisfação. E então Falls e Jonas dormiram juntos. Não apenas literalmente desta vez, se é que me entendem.
