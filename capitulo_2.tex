\chapter{555 95472}

Em seu primeiro dia, ele foi acompanhado pelo dono da empresa que ia lhe mostrando o lugar. O que ele faz lá é definido basicamente por George como ``digitoperaprogramalista''. Eles desempenham várias funções relacionadas a computadores como digitação, operação, programação e assistência técnica quando necessário. Seu chefe o levou através de salas bem iluminadas até a mais escondida das salas onde havia duas mesas. \mudanca{Um monte de entulho preenchia as paredes com teclados}, monitores e outras coisas. Em cada mesa havia um computador.

Apenas uma das mesas estava ocupada\mudanca{,} por um sujeito usando óculos e digitando insanamente\mudanca{,} em uma velocidade que Jonas sabia não conseguir chegar nem mesmo perto.

--- Este é George ---  o chefe disse o apresentando ao sujeito. --- O presidente da Terra da Fantasia.

Sem tirar os olhos da tela e ainda digitando, George soltou seu som de desprezo:

--- Tshh --- e voltando sua atenção para ele, discursou. --- Você sabe que isto está errado, não sabe Windows? A maioria das ambientações de fantasia possuem como sistema governamental a monarquia. Logo eu seria um primeiro-ministro ou mesmo um rei --- então ele levantou os olhos e me cumprimentou como se faz em Star Trek, depois voltou sua atenção para a tela do computador.

Depois de receber algumas instruções gerais, Jonas ficou sozinho com George, o que o deixou um pouco assustado.

--- Você sabe por que eu o chamo de Windows? --- George lhe perguntou.

--- Não.

--- Por que ele é um idiota --- George soltou, fazendo Jonas rir.--- E meu tio --- completou em tom sério como se o censurasse por ter dado risada.

--- Eu não sabia --- falou Jonas engolindo em seco.

--- Estou dizendo isso pois todo mundo acha fácil nosso trabalho. Que eu estou aqui só por ser o sobrinho do Poderoso Chefão. Por isso eu tento manter um padrão de qualidade, espero que você me ajude com isso.

Jonas fez um sinal afirmativo com a cabeça.

--- Como é o seu nome mesmo?

--- Jonas Arcádio da Silva --- falou.

--- Sou George\mudanca{,} como George Harrison, dos Beatles. Você tem algum tipo de apelido?

--- Não.

--- Hum --- George disse balançando a cabeça enquanto segurava o queixo. --- Vou te chamar de Jota.

--- Tudo bem --- respondeu Jonas achando graça da situação. Ele realmente precisava de um apelido?

--- Então\ldots\,Jota, que sistema operacional você usa em casa?

--- Meu computador está no conserto e\ldots

--- PORRA! --- George o interrompeu elevando o tom de voz. --- Você está brincando, não está? --- concluiu, rindo nervosamente

--- Não. Por quê? --- sem entender o motivo da súbita explosão de George.

--- Sem computador eu morreria tão rápido quanto sem água. Eu tenho três computadores para evitar situações como esta. Uso múltiplas distribuições Linux. Quer saber? --- ele disse apontando para Jonas seu dedo e depois fazendo um estalo. Ele se abaixou debaixo de sua mesa e me mostrou uma bolsa preta. --- Vou lhe emprestar um deles. Estou com ele aqui\ldots

--- Olha, obrigado. Mas não precisa.

--- ``Não'' não é uma opção Jota. Além do mais se você não aceitar, você vai me magoar.

Houve um pequeno silêncio em que eles se olharam. Constrangido, Jonas resolveu aceitar simplesmente para se livrar do assunto.

--- Tudo bem. Mas você nem me conhece, por que confiaria em mim?

--- Para confiar basta dar o primeiro passo, não? A partir de agora nós somos amigos --- disse voltando sua atenção para a tela do computador e digitando alguma coisa. --- Emprestar o computador é como dividir a namorada ou fazer sexo.

Jonas simplesmente fingiu não ter ouvido aquilo. Com o tempo eles se tornaram grandes amigos. Hoje, George é o melhor amigo de Jonas. Talvez o único, mas mesmo assim ele não tem do que reclamar. Ele é do tipo de pessoa que não faz questão nenhuma de fileiras e fileiras de amigos. Na verdade, sempre desconfiou que essas pessoas que são amigas de todo mundo na verdade não possuem amigo nenhum já que diluem sua atenção entre dúzias de outros.

Sete anos depois lá estava Jonas pensando se ligaria para a Novo Caminho\mudanca{, se} realmente teria coragem de fazer a \textsc{eqm}.

Havia alguma chance dele morrer. Será que ele estava pronto para morrer? A ideia o aterrorizava. Ele já pensou em suicídio. Mas ele nunca teve coragem o suficiente para fazê-lo. Não até aquele momento. Mas a ideia sempre estava por perto, como uma porta de emergência. \mudanca{Suicídio é violento} demais para sua personalidade. Uma vez um sujeito que trabalhava na Lethe se suicidou. A maioria das pessoas nem sabiam que ele trabalhava lá. Este talvez foi\mudanca{, obviamente,} um dos motivos que levaram o pobre a tirar a vida.

Jonas não o conhecia\mudanca{, mas} este gesto de tirar a própria vida atraiu-o misteriosamente. Era como se eles pudessem ter sido amigos. Toda a empresa apareceu para o enterro e todos se perguntavam nas rodinhas de conversa o que teria motivado o sujeito a tomar medida tão desesperada, \mudanca{lamentavam} o fato dele não ter deixado nenhum bilhete, carta, telefonema, nada que justificasse seus atos.

Albert Camus abriu um de seus ensaios mais conhecidos, ``O Mito de Sísifo'', declarando que só existe um problema filosófico realmente sério: é o suicídio. Julgar se a vida vale ou não a pena ser vivida é responder à questão fundamental da filosofia. O resto, se o mundo têm três dimensões, se o espírito tem nove ou doze categorias, se você vai escolher um sanduíche de presunto ou atum, aparece em seguida. São jogos.

Jonas não entendia por que eles tinham aquela necessidade que ele explicasse por que tirou sua vida. Jonas pensava que uma pessoa não precisa de um bom motivo para se matar. Basta para isso dar uma boa olhada em volta. Ver o quão fácil seria desistir de uma montanha de problemas, muitos dos quais você não tem controle algum, embora saiba fingir muito bem que tenha. ``Na verdade'', reflete ele, ``nós precisamos é de um bom motivo para continuar vivendo. Um bom motivo para nos fazer abrir os olhos todas as manhãs e fazer o que quer que façamos de dia''.

--- Isso te traz más lembranças, não? --- perguntou George.

--- Eu não quero falar sobre isso, George.

--- Não leve isso pelo lado pessoal, mas você é um chorão.

Jonas não sabia dizer se tinha um bom motivo para continuar vivendo. Ele queria respostas. Perguntas não paravam de ecoar em sua cabeça quando ele a colocava no travesseiro à noite. Talvez Jonas simplesmente se agarra tanto à vida porque \mudanca{sentia} como se ela fosse incompleta, faltando algo que ele não sabe o que é, nem se um dia ele já teve isso. Segue em frente sem saber direito para onde está indo. De qualquer forma, ele continuava.

Uma janela de mensagem instantânea pulou na tela de Jonas, \mudanca{tirando-o} de seus pensamentos. Era George. Ou melhor, era a sua personalidade na internet, Rev.~Voynich, que mandou a seguinte mensagem: ``Sabe o que significa Hipopotomonstrosesquipedaliofobia?''.

--- O quê? --- Jonas perguntou olhando para George que estava sentado em uma mesa, a no máximo dois metros dele.

George levantou a cabeça e apertou alguma tecla e sorriu para Jonas se espreguiçando. Tão logo fez isso, Jonas recebeu outra mensagem.

--- Claro --- sussurrou Jonas.

E lá estava a mensagem dele: ``Hipopotomonstrosesquipedaliofobia é uma doença psicológica que se caracteriza pelo medo irracional, fobia, de pronunciar-se palavras grandes ou complicadas. Se caracteriza pela aversão ou nervosismo em momentos nos quais o indivíduo deve empregar palavras longas ou de uso pouco comum (discussões técnicas, médicas, científicas etc.), assim como evitar ou não mencionar palavras estranhas ao vocabulário coloquial''.

--- E daí? --- eu perguntei a ele. Ele me olhou e começou a digitar.

''Você é um imbecil?'' George perguntou pelo mensageiro instantâneo para Jonas. ``Não,'' respondeu ele, ``imbecis são pessoas que preferem se comunicar pela internet do que pela vida real''.

--- Quer um spoiler sobre sua vida, George? --- Jonas perguntou, sendo ignorado. --- Você vai morrer sozinho.

--- Quer saber outro spoiler? ---  George retrucou. --- Brad Pitt é apenas um produto da imaginação.

Jonas olhou para George digitando no computador.

--- Você está digitando mais uma mensagem? --- questionou Jonas.

--- Não! --- replicou George como se tivesse ouvido um absurdo, como se estivesse sendo ofendido.

--- Não minta --- Jonas disse.

--- Não estou mentindo --- George disse em tom zombatório.

--- Por que aqui no mensageiro está dizendo que ``Rev.~Voynich está digitando uma mensagem'' --- Jonas falou.

Depois disso apareceu mais uma mensagem: ``A própria palavra Hipopotomonstrosesquipedaliofobia representa certa ironia, visto que, além de ser longa e estranha, indica uma fobia à palavras semelhantes''.

--- Sabe, George. Você devia tentar ter uma vida. Ouvi dizer que dá certo com algumas pessoas --- Jonas disse a George.

--- Eu tenho uma vida --- ele disse convicto, depois acrescentando: --- No Second Life.

--- Eu não gosto desse jogo --- Jonas falou desdenhando.

--- Jogo? --- George perguntou. --- O Second Life não é um jogo. Não há vencedores \mudanca{nem} perdedores lá.

--- Ah, não --- Jonas disse. --- Há muitos perdedores lá, George. Você precisa de uma namorada.

Logo, Jonas completou refletindo:

--- Eu preciso de uma namorada.

--- Você não faz o meu tipo, se é isso que está pensando --- disse George em tom sério.

--- Não, George. Não é isto que eu estava pensando.

--- Eu li em uma revista que\ldots

--- Que revista?

--- Irrelevante. Como eu dizia, eu li em uma revista que a maioria dos casais são formados com parceiros do círculo social das pessoas. Por exemplo, é alguém do trabalho, ou amigo de alguém do trabalho. Ou é amigo de um amigo. Você só encontrará o homem ideal explorando os círculos sociais.

--- O homem ideal? --- Jonas questionou rindo.

--- A mulher ideal --- corrigiu-se George.

--- Você disse o homem ideal --- corrigiu a correção de George.

--- Tshh ---  George desdenhou. --- Eu não disse isso.

--- Disse sim --- insistiu Jonas.

--- Deve ter sido por que era assim que estava escrito na revista. De qualquer forma, é irrelevante --- George concluiu voltando a sua atenção para o computador.

--- Eu imagino o tipo de revista que você lê --- Jonas murmurou sem atrair a atenção de George, tamborilou os dedos na mesa e se decidiu a ligar. Pegou novamente o folheto na sua gaveta. Lá estava o número de telefone da Novo Caminho: 555~95472. Ele teclou lentamente cada um dos números e esperou pela chamada. Afinal o que ele iria dizer? O medo tornou-se uma presença real \mudanca{enquanto} ouvia os sons de chamada. Pensou em desligar, quando uma linda voz surgiu do outro lado:

--- Novo Caminho, bom dia. Em que posso ajudá-lo?

--- Bom dia --- ele disse com certa dificuldade. --- Eu estou ligando sobre\ldots

--- O procedimento? --- a linda voz perguntou.

--- Isso. Exato --- ele falou. Seria muito mais fácil se ela continuasse guiando a conversa desse jeito.

--- Eu posso marcar uma consulta com um de nossos médicos. Têm algum dia ou horários de preferência?

--- Pode ser qualquer dia entre as treze e as quinze horas --- falou Jonas.

--- Na próxima segunda-feira, treze e meia?

--- Tudo bem.

--- Qual o nome do senhor?

--- Jonas. Jonas Arcádio da Silva.

--- Está marcado para segunda-feira às treze e meia então.

--- Obrigado --- ele falou.

--- Obrigado e tenha um bom dia --- a linda voz falou.

--- Para você também --- ele disse, enquanto do outro lado a ligação era encerrada.

Estava feito. Após o final de semana ele iria descobrir se ele teria mesmo coragem de se arriscar a fazer uma experiência de quase-morte. Não foi tão difícil. Um passo ridículo para a humanidade, um grande passo para Jonas.

--- Você acredita em Céu e Inferno? --- Jonas perguntou para George quando não tinha nada para fazer algum tempo depois de ter feito a ligação.

--- Tshh ---  George desdenhou da pergunta enquanto digitava. --- Eu sou niilista, Jonas. Eu não acredito em nada.

--- Tudo isso é nada? --- Jonas perguntou mostrando a sala ao redor dos dois. --- Então é nada o mundo todo, e tudo o que nele há, o céu é nada, a internet é nada, são nada também todos esses nada?

--- Tudo é igualmente sem sentido --- George disse, procurando corrigi-lo.

--- Você não acredita em paz, justiça ou alienígenas? --- Jonas perguntou para o provocar.

--- A chance de alienígenas existirem é relativamente alta. Há bilhões de estrelas somente em nossa \mudanca{galáxia}. Milhares delas possivelmente com planetas girando ao redor. Alguns deles a uma distância que permita a vida, como aqui. Logo, mesmo que as chances de surgir vida sejam bem baixas tendo os planetas que atender a uma série de requisitos, ainda assim há chance de haver vida extraterrestre em pelo menos alguns deles. Com os mais de quinze bilhões do Universo é quase certo que já tenha surgido alguma civilização em outro lugar. Sobre paz ou justiça eu não possuo fatos que suportem ou descartem a existência das mesmas.

--- Se é quase certo, onde estão eles?

--- Há várias linhas de pensamento Jonas, você não entenderia.

--- Como assim? Você não deve saber --- Jonas disse para atingir o colossal ego de George.

--- Eu não quero perder meu tempo, só isso. Eu estou ocupado\ldots

Jonas se esticou e viu a tela de George, que percebendo isso fechou a tela, mas tarde demais.

--- Você está jogando paciência! --- Jonas o acusou.

--- E graças a você eu perdi a minha\ldots

--- Vamos, me diga duas dessas ``diversas'' linhas de pensamento --- Jonas o provocou desenhando as aspas no ar.

--- Um: a civilização extraterrestre evoluiu até um ponto em que se depararam com algum tipo de energia ou tecnologia que levou ao fim deles. Como ainda pode acontecer conosco graças a Einstein e a bomba nuclear. Dois: eles estão jogando.

--- Jogando?

--- Games, Jonas. \textsc{mmoprg}, essas coisas.

--- Por quê?

--- Viagens espaciais são caras, possivelmente demoradas e arriscadas. Eles podem ter ficado em casa se divertindo enquanto enviaram sondas por aí. Talvez essas sondas deram origem a outras espécies. Como nós. Ia ser muito divertido se descobrissem que Deus não dá a mínima para nós simplesmente por que tudo o que ele quer é ouro e pontos de experiência.

George então voltou seu rosto como se houvesse se esquecido de dizer algo importante e acrescentou:

--- Há também a possibilidade deles serem como os alienígenas de \mudanca{\emph{Sinais}} e não nos invadirem por que poderiam ser destruídos com água.

--- Esse filme é ridículo.

--- Você não vai falar mal de algum filme de Night Shyamalan na minha frente ---  George falou\mudanca{,} em posição de ataque intelectual.

--- Eu já estou fazendo isso George. O filme é ridículo. Água ser o ponto fraco dos alienígenas? Esse planeta é três quartos água. Deve ser a espécie alienígena mais idiota da história do cinema.

--- Isso por que você não foi além da superfície, Padawan. Eles eram parte de uma raça que perderam seu planeta natal por algum desastre natural e que vagaram anos e anos, talvez séculos, atrás de um planeta que oferecesse condições para a vida deles. A Terra foi um candidato, talvez o único encontrado. Havia esse inconveniente mas eles não tinham para onde ir. Arriscaram e foram derrotados.

--- Você inventou tudo isso --- Jonas vaticinou.

--- São os fatos --- George disse com desdém.

--- Você acredita nessa sua teoria? Eu duvido que o Night Sei-Lá-O-Que pensou nisso.

--- Nem você. Mas eu acredito que essa teoria esteja correta.

--- Ah, você acredita? Porque há um minuto atrás você disse que era um niilista e não acreditava em nada.

--- Irrelevante para o assunto --- ele disse.

Quando George voltou a trabalhar ou jogar paciência no computador, Jonas escreveu em um post-it a data de sua consulta e colou no monitor.

--- Já sabe com que fantasia você vai? --- George perguntou.

--- Não. Não faço a mínima ideia, e você?

--- Eu tive a ideia para a fantasia que eu vou usar dois segundos depois de sair da última festa --- ele disse.

--- E o que é?

--- Eu não vou estragar a surpresa. Seria spoiler --- George falou abrindo um de seus sorrisos que faziam Jonas agradecer que ele extravasasse sua raiva em games de tiros de primeira pessoa, criando e destruindo civilizações e em corridas de karts e não saindo por aí e destruindo tudo. Um pensamento que o acompanhava desde o primeiro dia.

--- Se você fosse um serial killer, qual você seria? --- George o perguntou justo no segundo dia, fazendo com que Jonas passasse a ficar meio paranoico até descobrir que George era um ser humano afinal.

--- Eu não sei. Não conheço muitos --- Jonas disse tentando fazer uma piada, mas depois acrescentando por medo: --- Mas, não que eu queira ser apresentado a nenhum.

--- Eu seria um original --- disse George. --- Usaria uma arma de gelo pois a arma do crime derrete e desaparece. Sem impressões digitais. Nunca me descobririam.

Jonas olhou para George e arqueou as sobrancelhas, enquanto pensava: ``Se algum dia aparecerem pessoas assassinadas sem se saber a arma ou for algo de gelo, eu trabalhei com um serial killer''.

--- Por que está me olhando? --- George perguntou incomodado.

--- Nada, nada não --- Jonas respondeu voltando a face para o computador, depois para George.

No final acabaram se tornando amigos. George não é do tipo que fala muito sobre si mesmo embora possa passar horas falando de \emph{Star Wars}, \emph{Senhor dos Anéis}, \emph{Matrix} e outros. Tinha centenas de \textsc{dvd}s e todos os consoles. Eles se divertiam muito. Quando pensa nos primeiros dias a forma como ele via George antes de conhecê-lo e como o via hoje, como seu grande amigo, seu melhor amigo, Jonas pensava no quão nossos julgamentos podem nos afastar de pessoas que podem ser extraordinárias.

É tudo uma questão de ponto de vista. É muito saudável se colocar no lugar das outras pessoas. Faz você descobrir ângulos incríveis.
