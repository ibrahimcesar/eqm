\chapter{O que você faria se a pessoa que você mais ama morresse?}

Quem já amou que não fosse à primeira vista?

Tomás nunca fora muito romântico. Se você perguntasse \mudanca{o que} ele pensava a respeito de amor à primeira vista\mudanca{,} talvez ele simplesmente risse de você, como se a opinião dele fosse o senso comum a respeito do assunto.

Mas ele esteve todo esse tempo errado. Pelo menos foi assim que ele sentiu após pousar seus olhos sobre Ana. Quando ele a viu em uma festa, foi como se o mundo tivesse parado. Ele não conseguiu tirar os olhos dela. Como poderia? Não é todo dia que se encontra a pessoa que em seu íntimo você sabe que é aquela predestinada a ser sua.

Ele não sabia o que fazer. Suas mãos começaram a suar e ele apenas concordava com tudo o que lhe diziam. Ela acabou indo embora da festa e não tomou a iniciativa, não teve coragem nem de lhe dirigir a palavra. Quando ela foi embora ele foi tomado por uma grande sensação de vazio.

Começou a fazer de tudo para se aproximar dela. Até que um dia a chamou para assistir um filme e ela aceitou. Dois anos depois eles estavam se casando. Sempre que pensava na mulher se lembrava dela naquele dia, com seu vestido vermelho\mudanca{,} e sentia o mesmo sentimento que parece retirar todas as nossas forças e dominar os pensamentos. Mas agora, com os anos de rotina e casamento, ele já não parecia ser metade do homem que costumava ser.

Não fazia mais elogios, simplesmente por não saber o que ela iria pensar dele. Quando estava trabalhando\mudanca{,} tudo o que ele pensava era em voltar para casa e estar com a esposa. Mas assim que botava os pés em casa, não se sentia \mudanca{confortável}. Não queria estar perto dela e sempre inventava uma ou outra desculpa para estar em outro cômodo. O que estava errado com ele? Sempre que seu chefe queria que alguém fizesse hora extra ou ficasse mais tarde, ele não pensava duas vezes em se oferecer.

Ele era um dos melhores vendedores de carro de toda a cidade. Conhecia todos clientes pelo nome e sempre estava com um sorriso no rosto. Era até mesmo berrante a diferença entre esse homem trabalhando e em sua casa.

--- É o carro perfeito para a senhora, Sra.~Robinson.

--- Tudo bem, Tomás eu vou acreditar em você. Porque se está mentindo\mudanca{,} o senhor o fez de modo exemplar.

--- O senhor não irá se arrepender de comprar este carro.

--- Espero que não.

--- E ainda vai estar ajudando a salvar o planeta, Sr.~Arcádio! Isso não é ótimo? --- ele dizia todo sorriso.

---  Eu posso fazer um bom desconto para você, Sr.~Burns.

---  Desconto? E quanto isso vai me custar?

Ele amava a esposa, \mudanca{nem} sequer pensava em traí-la, mas se sentia mal, com o olhar distante. Os anos foram se acumulando e a tristeza, que um dia ele pensou apenas ser temporária, se tornou a regra em sua vida. Até que um dia ele voltou para casa, após mais um dia vendendo carros e encontrou a casa vazia. Ele sempre chegava e logo jantava. Ficou preocupado, mas não deu muita atenção. Foi tirar a roupa e tomar uma ducha.

Enquanto ele tomava banho, o telefone tocou algumas vezes. Como ele estava no piloto automático, nem se importou em ir atender, já que nunca atendia um telefone em sua casa. Ana sempre fazia isso. Mais tarde quando finalmente atendeu o telefone, e Regina lhe disse o que acontecera, ele se deu conta de que desperdiçou toda uma vida.

Passou dias chorando. Seu chefe resolveu lhe dar férias, já que era um vendedor exemplar e a morte da esposa havia o afetado de forma colossal. Ele se culpava por nem ao menos ser capaz de dizer a última vez que havia abraçado a esposa, ou lhe dado um beijo. Ou mesmo a chamado de ``querida''.

Regina passou a levá-lo em cultos, enquanto ele discutia com membros da família a respeito dessa tal de Novo Caminho. Já havia ligado várias vezes lá e falado com uma mulher que lhe dizia um monte de desculpas esfarrapadas. Chegou a visitar dois advogados e os dois diziam a mesma coisa. Não havia como apelar. Ela havia se submetido voluntariamente e em sã consciência a um procedimento que sabia ser perigoso, possivelmente mortal. Ela sabia dos riscos.

Em um dos cultos, Tomás se pegou pensando que era melhor que toda essa coisa de céu fosse verdade mesmo, ou Ana ficaria realmente brava. Nesse dia ele ouviu o pastor fazer um discurso que o tocou. Ele sentiu o dedo de Deus.

--- O caminho do homem de bem é cercado de todos os lados pelas iniquidades do egoísmo e tirania dos homens maus. Abençoados os que, em nome da caridade e da boa vontade, conduzem os fracos pelo vale das sombras, pois ele é o guardião de seu irmão e o que encontra os filhos perdidos. E executarei neles grandes vinganças, com castigos de furor, e saberão que eu sou o Senhor, quando eu tiver exercido a minha vingança sobre eles. Ezequiel, capítulo vinte e cinco, versículo dezessete --- foi como o pastor encerrou o sermão sobre as cruzadas espirituais, o dever de todo cristão.

Ele sabia o que tinha que fazer. No dia seguinte, quando colocou seus pés nos tijolos amarelos que\mudanca{,} através de um jardim, levam até a porta automática da Novo Caminho, Tomás sentia que estava fazendo a coisa certa, não podia se calar. Em uma das mãos ele carregava uma Bíblia, que estava lhe confortando desde a morte da mulher.

--- Onde está o maldito que fez isso com minha esposa, o Satanás. Onde está Satã?

--- O quê? --- perguntou a atendente\mudanca{,} levantando\mudanca{-se} e bloqueando sua frente.

Edgar que estava do lado dela, assinando um papel, o olhou e imaginou saber de quem se tratava.

--- Você é Tomás de Torquemada? Marido de Ana de Torquemada?

Ele apenas afirmou, visivelmente muito irritado. De um dos corredores, surgiu um estagiário com olhos curiosos.

--- Tomás, se acalme. Venha ao meu escritório para conversarmos --- Edgar disse o tocando no braço. --- Júlio, pode pegar um copo de água?

Tomás repeliu o toque de Edgar com violência.

--- E você, quem é?

--- Sou Edgar. O psicólogo da Novo Caminho.

--- Então é você que foi mandado para dar um jeito em mim? Me mandar à merda de uma forma que eu queira ir. É isso?

--- Tomás, eu converso com cada pessoa que passa pelo procedimento. Com sua mulher não foi diferente. Nós gravamos a consulta por segurança. Se o senhor me acompanhar posso ajudá-lo a entender a decisão de sua mulher\ldots

Tomás procurou se acalmar. Ele estava com raiva e não queria de forma nenhuma se submeter ao que quer que eles tivessem para dizer, mas ele também não poderia deixar de lado esta oportunidade. Descobrir que Ana havia feito tudo isso por suas costas e sem nem ao menos alertá-lo era uma das coisas que mais o enfureciam. Mas o fazia se sentir impotente.

``Eu irei gravar nossa conversa, tudo bem?''\mudanca{,} a voz de Edgar perguntava na gravação que ele rodava para Tomás.

``Sim'', a voz de Ana surgiu na sala. Tomás sentiu seu coração apertar.

`Você está consciente dos riscos envolvidos no procedimento que quer fazer?''

``Sim.''

``Você declara estar realizando-o por livre e espontânea vontade?''

``Sim, declaro.''

``Seu contato pessoal é Regina, sua irmã. Quer deixar registrado alguns palavras para ela caso o pior aconteça?''

``Eu não sei se ela vai entender por que eu estou fazendo isso. Mas diga a ela que a amava muito.''

``Você é casada, senhora Torquemada?''

``Sim.''

``Não quer deixar nenhuma mensagem para seu marido?''

``Não. Vai demorar algum tempo para ele perceber que eu morri. Eu sou invisível para ele. Quando estou em um lugar, é o mesmo que não estar lá. Eu sou apenas um móvel empoeirado. Quero me sentir viva de novo.''

Tomás fechou os olhos e apreciou o impacto que aquelas palavras fizeram nele. Ela ecoaram e ecoaram em sua cabeça.

--- Quem é o responsável por isso? --- Tomás disse se levantando.

--- O Dr.~Roberto Mouir. Temo que ele não poderá atendê-lo.

Tomás saiu da sala de Edgar, que o seguiu. Tomás tomado de fúria abriu uma porta, onde não havia ninguém. Partiu para abrir outra quando Edgar o bloqueou. POW! Tomás aplicou um soco no rosto de Edgar que caiu no chão. Da sala~Tlön, saíram dois estagiários que olharam para ele, aterrorizados. Edgar se levantava.

--- Chamem a polícia --- Edgar murmurou\mudanca{,} colocando a mão nos olhos.

Tomás ficou com medo de ser preso e começou a sair, a passos rápidos, sem olhar para trás. Ele fechava os olhos e as palavras estavam lá, ecoando. Um loop~sentimental.

A polícia acabou indo até a casa dele. Ele foi levado para a delegacia e prestou depoimento. Sua fiança foi paga pelo Pastor~FFF.

--- Você agiu mal, não vou mentir --- o pastor lhe disse\mudanca{,} quando saíram da delegacia.

--- Eu não posso simplesmente ficar em casa, pastor. Eu sinto que preciso fazer algo. Qualquer coisa. Se eu ficasse em casa, toda essa dor me devoraria. Eu tenho que fazer algo!

--- Mas eu acho que posso ajudá-lo. A Novo Caminho está iludindo as pessoas. Pelo que eu pesquisei o túnel de luz que as pessoas encontram nessas experiências de quase-morte é Satã, que no momento de mais fraqueza ilude as pessoas. Vou organizar uma manifestação, Tomás. Nós vamos fazer eles fecharem. Eu lhe prometo.

Antes displicente em suas visitas à igreja, Tomás passou a ser dedicado e ia a todos os cultos. Esperava com ansiedade o dia da manifestação. Passava todos os dias em frente à Novo Caminho desejando cruzar com o tal Dr.~Roberto. Ele então começou a parar o carro lá em frente. E ficava apenas observando. Ele tinha que rezar e rezar muito para não sair de dentro de seu carro e fazer alguma besteira. Até que um dia ele viu um rapaz descendo de seu carro. Ele não pode se controlar, ele respirou fundo e começou a caminhar.

--- Onde está Satã? Onde está Satã? --- ele entrou falando.

Ele viu a recepcionista o olhando preocupada e abrindo a gaveta. Ele continuou andando.

--- Sr.~Tomás, por favor, afaste-se --- a moça lhe disse.

--- Eu vim conversar com aquele sujeito --- Tomás falou. --- Você não têm nada a ver com isso.

Então ele continuou andando, passaria por ela se fosse necessário. Deus estava do seu lado.

--- Pare! --- a moça lhe\mudanca{,} disse espirrando algo nos olhos. Ele caiu na mesma hora no chão. Ardia muito, seus olhos lacrimejavam em profusão, começou a tossir, sentindo a garganta irritada e os pulmões pegando fogo.

``Deus! Deus!'', se contorcia ele no chão.

--- Meus olhos! --- disse uma\mudanca{,} voz perto dele.

--- O que aconteceu? --- \mudanca{Tomás ouviu} a voz de Edgar.

--- Tomás de novo --- alguém disse.

--- Já chamei a polícia.

--- Levem a Falls e aquele cliente para passarem água no rosto.

--- Por que você insiste? ---  Edgar perguntou\mudanca{,} perto de seu ouvido. --- Eu sei que provavelmente vai ignorar tudo o que eu disser, porque você obviamente não está raciocinando. Está se deixando levar apenas pela emoção bruta, incontrolável, mas isso não vai trazê-la de volta. Sei que se consome de culpa, mas você deve seguir em frente e não ficar preso a isso. Entende?

Por pouco que\mudanca{,} desta vez, a polícia não \mudanca{permitiu} que ele \mudanca{fosse} solto novamente sob fiança. O pastor~FFF conversou um tempo com o delegado. Ele saiu com a condição de não passar mais perto da Novo Caminho\mudanca{. Caso} estivesse nos arredores\mudanca{,} ele seria preso sem ter direito a fiança. O pastor deu sua palavra que o supervisionaria.

--- Eu tive que assinar um termo de responsabilidade. Eu sei que você não vai me decepcionar --- lhe disse o pastor.

Somente depois de ponderar muito e fazê-lo prometer não fazer nada, que o Pastor~FFF, permitiu que Tomás acompanhasse a primeira manifestação. Foram apenas uns poucos gatos pingados e logo a polícia chegou. O Pastor~FFF foi conversar com eles.

Voltou dizendo que deveriam dar a manifestação por encerrada.

--- Por quê? ---  perguntou Tomás com cólera\mudanca{,} quando foi tirar satisfações.

--- Nós não vamos ter problemas com a polícia --- deixou bem claro FFF.

--- Se curvando à lei dos homens quando há a lei divina, pastor?

--- Estou fazendo apenas o que é o certo. Apenas isso. Nós voltamos outro dia, com mais pessoas, mais organizados. Além do mais\ldots\,Você nem mesmo devia estar aqui. Quer ser preso, Tomás?

As palavras de FFF foram cumpridas ao pé da letra. Quase dois meses mais tarde, uma grande multidão se reunia ao lado de fora da Novo Caminho protestando contra o ato vil e diabólico que eles faziam. Havia até mesmo um cartaz com a foto de Ana.

Tomás viu a multidão crescendo, e no meio dela teve uma ideia que o fez correr ao seu carro. O Pastor~FFF o viu se afastando\mudanca{,} mas não tinha tempo para falar com ele\mudanca{. Talvez} fosse melhor que ele não estivesse ali para causar problemas.

O que o pastor não tinha ideia era que\mudanca{,} meia hora mais tarde, Tomás estacionava novamente, desta vez com uma arma na cintura\mudanca{. Seu} irmão a havia dado a ele uma vez. Nunca havia disparado um tiro sequer, mas sempre há uma primeira vez, não é o que dizem? Deus estava do seu lado.

--- Tomás! --- ele escutou sua cunhada lhe chamar, mas não tinha tempo. Ele precisava fazer o acerto de contas. Seria a mão de Deus fazendo a justiça divina.

--- Temos que mostrar a ira divina para eles --- Tomás gritou para todos, imbuído de um sentimento de estar fazendo a coisa certa.

--- Não!--- O Pastor~FFF falou com todos. --- Estão se esquecendo do que Jesus disse? --- ele falou\mudanca{,} balançando a Bíblia em sua mão.

--- Sim! --- retrucou Tomás. --- Ele disse: ``Não cuideis que vim trazer a paz à terra; não vim trazer paz, mas a espada!''

Falou isso, com a certeza de que Deus estava com ele, de que estava fazendo a coisa certa. Então avançou.
