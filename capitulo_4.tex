\chapter{Cara legal}

Todos sempre perguntam a ela qual a origem de seu nome. É francês. Depois invariavelmente perguntam se seu pai ou sua mãe são franceses. Sua mãe só achou um nome bonito. O que o seu nome dizia sobre ela, se nem ao menos o escolheu?

Ela trabalhava na Novo Caminho como recepcionista desde a inauguração, há alguns anos. Ela gostava de trabalhar lá. Tirando alguns tipos bem estranhos que ela tinha que lidar, era um trabalho interessante.

Nunca tivera sorte no amor. Nem mesmo no jogo. De alguma forma\mudanca{,} parecia que o mundo conspirava contra ela, embora Paulo Coelho dissesse o contrário. Mas\mudanca{,} mesmo assim\mudanca{,} ela tentava não se deixar abater por essas coisas\mudanca{,} e era uma seguidora daquilo que ela chamava de Teoria Playmobil. Playmobil são pequenos bonecos, com mãos em forma de~``U'', que movem os braços e as pernas ao mesmo tempo, cabelo destacável da cabeça e um sorriso no rosto. Sempre um sorriso no rosto. Aconteça o acontecer. Seja a versão Velho Oeste, a Espacial, \mudanca{Policial,} ou qualquer outra. Eles poderiam estar no meio de uma guerra\mudanca{,} e lá estaria o sorriso estampado em suas caras de plástico. E esse é o espírito da Teoria Playmobil: aconteça o que acontecer, mantenha sempre um sorriso no rosto. Esta é a sua vida. Cada hora a mais é na verdade uma hora a menos.

Era nisso que ela pensava sempre que algo ruim acontecia ou acordava com uma pequena nuvem negra acima \mudanca{da} cabeça, o que não era lá muito difícil de acontecer. Mas se ela não conseguisse utilizar a Teoria Playmobil e ficasse irritada ou triste e discutisse sem motivo\mudanca{, sempre} teria à mão a desculpa da TPM. Ela poderia ter três ou quatro TPMs em um mesmo mês. Nunca era questionada por homem nenhum, isso os desarmava. Parece até que eles não sabem contar.

Falls adorava dançar, por isso\mudanca{,} quando uma colega de faculdade a convidou para ir na Festa do Fim do Mundo\mudanca{,} ela não pensou duas vezes. Não que estivesse afim de arranjar alguma companhia ou \mudanca{o tradicional relacionamento descartável, que todo mundo busca nessas festas, que dura meia-hora, ou na, maioria das vezes, muito menos}. Ela já se cansou dos cafajestes. Só queria ir dançar. Mas sua colega fez uma ressalva:

--- Vista algo decente.

--- Tudo bem, Regina --- Falls respondeu\mudanca{,} sem saber ao certo se isso era apenas um conselho ou uma repreensão pela roupa que Regina imaginava que ela iria vestir. --- Que fantasia você vai?

--- Eu vou de gata. O animal.

--- Você têm uma gata\mudanca{,} não é mesmo?

--- Freya é o nome dela. É uma gata persa. Você têm algum gato?

--- Eu sou mais uma pessoa de cachorros --- Falls disse\mudanca{,} meio sem jeito.

Regina fez uma cara de aversão. Falls ficou pensando se poderia dividir as pessoas entre esses dois tipos básicos. Pessoas de gatos e pessoas de cachorros. Gatos são folgados, dorminhocos e abandonam a casa do dono para dar passeios sabe-se lá por onde. Eles precisam ser mimados o tempo todo. Cachorros tentam ter a atenção de seus donos o tempo todo, vigiam a casa e fazem sexo com a perna das visitas. Aliás, cachorros salvam pessoas de serem afogadas. Pelo menos Falls lembra-se de ter visto isto em um filme. Se você estiver se afogando e for dono de um gato\mudanca{,} talvez ele só perceba que você morreu quando o prato deles continuar vazio após três dias. Para resumir: quem é uma pessoa de gatos quer servir. Quem tem cachorros quer ser obedecido.

Do que ela iria vestida? Não tinha ideia nenhuma\mudanca{,} embora obrigatoriamente devesse ser algo comportado. Com isso\mudanca{,} ela não teria problema\mudanca{s}. Embora tivesse um corpo do qual se orgulhasse muito, com exceção de ``estrias na coxa e os dedos do pé'', Falls não gostava de usar roupas muito curtas. Davam-lhe a sensação de estar nua. E ela sabe que ao usar tais roupas é exatamente assim que se está na imaginação de cada homem que cruza com você.

Ela não teve tempo de pensar no que vestir pois teve que correr para chegar na Novo Caminho. Ela fazia a faculdade pelas manhãs e trabalhava à tarde. A filha do chefe, Dr.~Roberto, cursava o colégio de manhã, como não gostava de ficar sozinha em casa, passava a maioria de suas tardes na sala de espera\mudanca{, o que} acabou fazendo que elas duas se aproximassem e de certa forma fossem até mesmo amigas.

--- Com o que acha que eu devo ir, Sarah?

--- Uhm\ldots\,--- disse Sarah. --- Do que você gosta?

--- Na verdade, como a festa é hoje, eu vou aceitar qualquer coisa que conseguir achar.

Sarah sorriu e disse:

--- Já que é assim, meu tio tem uma fantasia em casa.

--- É mesmo? --- Falls perguntou com interesse. --- Vou falar com ele.

Quando teve uma oportunidade, Falls tocou no assunto.

--- Patrick?

--- Hã? --- ele perguntou quando estava voltando para seu consultório.

--- Eu vou ir em uma festa à fantasia e Sarah me disse que você tem um fantasia.

--- Não que eu me\ldots\,--- disse Patrick tentando se lembrar de alguma fantasia. Então ele começou a rir. --- Na verdade eu tenho, mas não acho que você vá ficar muito bem com ela.

--- Ah, eu não me incomodo --- Falls disse.

--- A barba pode pinicar um pouco --- Patrick observou rindo enquanto passava a mão em volta de seu rosto.

--- Barba?

--- A única fantasia que eu tenho, e Sarah sabe disso melhor que eu, é de Papai Noel.

Sarah gargalhava sozinha ao ver a expressão de Falls.

--- Mas eu sei que o Roberto guarda uma fantasia que era da Larissa. Não sei se ele vai se importar.

Falls \mudanca{sabia somente} que o Dr.~Roberto era viúvo, nada mais. Talvez Larissa fosse sua esposa. Roberto não tocava nesse assunto com ninguém. Muito menos com Falls.

--- Pode deixar que eu falo com ele --- Patrick falou antes de começar a andar. Ao passar perto de Sarah\mudanca{, ela} estendeu a mão para ele, \mudanca{deram} um toque olhando para Falls.

--- Falls Noel --- se divertia Sarah.

Dr.~Roberto a levou até a casa em que vivia ele, a filha e o cunhado. Falls ficou sem jeito de aceitar vestir a fantasia, ainda mais sendo de alguém já que morreu. \mudanca{Até} sentiu um calafrio.

--- Olhe como ficou bem em você --- disse Sarah olhando para Falls vestida.

--- Pode ficar com ela. Acho que Larissa \mudanca{a vestiu uma vez só}.

--- Eu não posso aceitar --- ela disse.

--- \mudanca{\emph{Aceite}} --- ele disse\mudanca{,} enfatizando bem a palavra. --- Além do mais, nós não precisamos dessas coisas. A temos em nosso coração. Se você têm isso, coisas não importam.

Falls sorriu e aceitou a fantasia, constrangida de ficar vestida de fada perto de seus chefes. Ela ficou jogando banco imobiliário com os três até Regina chegar para apanhá-la. O que, sinceramente, poderia ter demorado mais.

--- Que pena que minha carona chegou. Justo agora que meus aluguéis estavam começando a render.

--- Sinta-se convidada para jogar quando quiser --- falou Roberto.

--- Tenho certeza de que você vai arranjar um namorado hoje --- falou Sarah\mudanca{,} arrancando risada de todos.

``Eu tenho certeza que não'', Falls pensou mas apenas sorriu e se despediu deles. Ela riu ao ver Regina, sempre tão séria\mudanca{,} com um bigode desenhado nas bochechas e um pequeno triângulo de cabeça para baixo preto fazendo as vezes de nariz do gato. Usava também uma tiara com duas orelhas.

--- Não vai me dizer que está usando um rabo também --- Falls falou rindo. Regina fez uma risada contrariada, mas mesmo assim riu. Mais tarde, quando Regina desceu do carro, Falls descobriu que sim, ela estava usando um rabo.

--- E você é uma fada madrinha?

--- Foi o melhor que eu pude encontrar.

Regina surpreendeu Falls quando tomou seu terceiro copo de bebida. No inicio ela se recusou a beber, já que teria que dirigir. Mas depois acabou se rendendo. Estava falando mais do que normalmente e mesmo se permitindo dar risada. E uma risada alta e estridente, se querem saber. Falls não queria beber. Não por enquanto. Quando acompanhava Regina no bar ela tinha vontade de dizer ``Uma bebida e um amor sem gelo, por favor''.

Até que Regina misteriosamente desapareceu segurando na mão de um sujeito vestido com uma fantasia que Falls diria ser de pastor evangélico, ou no caso de Regina havia o grande risco de realmente ser um pastor evangélico. ```Eu espero ter uma carona'', ela pensou um pouco assustada.

No começo ela apenas ficou se balançando, depois ensaiou alguns passos.

Falls não conseguia dançar na primeira em vez em que saiu com suas amigas. Se sentia desengonçada e além do mais não tinha ideia nenhuma de como começar, o que fazer no meio e como terminar. Um peixe fora d'água.

--- Vamos dance! --- diziam suas amigas em seu ouvido, a fazendo corar.

--- Eu não sei dançar!

--- Quer saber um segredo? Ninguém sabe, mas no escuro da pista de dança qualquer coisa que você faça é dançar.

Ela não estava certa de que aquilo soava como verdade e ficou um pouco receosa. Ela era muito insegura naquele tempo, talvez por nunca ter saído antes ou por ser muito tímida.

--- No escuro todo mundo é bonito --- disse uma de suas amigas.

``Ela disse isso para eu ter cuidado ou ficar feliz?''\mudanca{,} pensou Falls. Então\mudanca{,} depois de um tempo em um estado intermediário entre dança e não-dança, Falls começou a se mexer. Aos poucos foi deixando os movimentos um pouco mecânicos e baseados na observação de suas amigas para fazer seus próprios passos. Ela não acreditava em si mesma. Foi nesse dia que Falls descobriu que tinha uma parte de seu corpo chamada pélvis.

Desde então ela passou a adorar dançar. Antes ela achava que se você ia para uma festa ou o que quer que fosse, era como se você estivesse indo lá \mudanca{porque estava} no grande mercado de relacionamentos. Ela simplesmente ignorou essa ideia anterior. Não que ela não estivesse atrás de um relacionamento, mas \mudanca{não gostava nem um} pouco dos métodos intrusivos de abordagens da maioria dos sujeitos. E ela queria um namorado, todos querem alguém para dividir as alegrias e as tristezas, não é mesmo? Um clichê, claro. Mas o mundo é um terrível clichê. Não sabia de onde ele viria, mas com certeza estava esperando. Enquanto isso não acontecia, ela saía apenas para dançar, curtir a si mesma. Isso lhe bastava.

