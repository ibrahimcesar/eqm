\chapter{Dance dance dance}

\vspace{3em}

Deve dançar. Enquanto a música estiver tocando, você deve continuar a dançar. Entende o que quero dizer? DANÇAR, CONTINUAR DANÇANDO. Não deve pensar no motivo \mudanca{nem} no sentido disso, pois eles praticamente não existem. Se ficar pensando nessas coisas, seu pé ficará imóvel. Uma vez parado, já não será capaz de agir. Por isso, não deve parar de mover os pés. Por mais que lhe pareça uma tolice, não deve ligar. Deve continuar dançando, dando os passos. Deve ir amolecendo, mesmo que aos poucos, tudo o que estava completamente rígido. Use tudo o que pude usar. Dê o máximo de si. Não há o que temer. Só lhe resta dançar. E dançar de modo exemplar. A ponto de todos ficarem admirados. Pois, assim, talvez eu consiga ajudá-lo. Por isso, dance enquanto a música estiver tocando. DANCE ENQUANTO A MÚSICA ESTIVER TOCANDO!!!

E Falls fez isso de modo exemplar.

Ela acordava no dia seguinte com as pernas todas doloridas, os ouvidos zumbindo e com os olhos querendo fechar todo o resto do dia, tornando impossível se concentrar em \mudanca{seu trabalho} ou no que quer que fosse. Mas ela sentia que fazia bem para sua auto estima. Ela era feliz.

Logo suas pernas começaram a lembrá-la de sua condição humana e ela foi desacelerando os movimentos até parar, suada, com dores na perna\mudanca{,} mas extremamente satisfeita consigo mesmo, nem se importando com três cantadas baratas que jogaram \mudanca{em} cima dela, e todas relacionadas a ela realizar desejos dos sujeitos. Ela se cansou e resolveu parar um pouco. Foi seu ponto de inflexão. Começou então a ter alguns pensamentos tristes, desses que temos\mudanca{, às vezes,} vindos sabe-se lá de onde, mas que invadem a nossa consciência sem serem convidados e lá se instalam. Tentando se esquivar desses pensamentos negativos, \mudanca{imaginou} como o mundo seria se fosse um musical, e\mudanca{, do nada,} as pessoas começassem a dançar como dançarinos profissionais e a cantar com vozes belíssimas.

Na verdade\mudanca{,} uma parte do mundo foi assim durante a Idade Média. A isso deu-se o nome de mania dançante. A mania dançante foi um fenômeno que ocorreu em sua maioria na porção continental da Europa entre os séculos quatorze e dezessete, na qual um grupo de pessoas dançavam pelas ruas das cidades, muitas vezes falando coisas sem nexo e não canções ensaiadas e escritas por músicos profissionais, até eles colapsarem pela exaustão. A primeira grande manifestação ocorreu em Aachen na Alemanha, em 24~de~Junho de~1374. Os dançarinos iam pelas ruas gritando loucas visões e continuavam a gritar\mudanca{,} mesmo depois de caírem exaustos. A mania se espalhou rapidamente pela França e Países Baixos. Ficou em cartaz por um bom tempo.

O pico da mania foi atingido em 1418 em Strasbourg. Em algum ponto, tantas pessoas estavam ou sendo afligidas com a mania dançante ou \mudanca{tentando} dar assistência\mudanca{,} ou talvez só aproveitando o espetáculo\mudanca{,} que a cidade parou totalmente.

Não se conhecem atualmente as causas, mas a teoria mais aceita é que foi tudo graças ao pão. Na verdade ao esporão do centeio, \emph{Claviceps purpurea}, um fungo parasita que ataca o centeio que \mudanca{se usava} para fazer tanto o pão como a cerveja. Então\mudanca{,} deve ter atacado muita gente mesmo. É um fungo conhecido por ser alucinógeno, e usado para fabricar \textsc{lsd}. Ela apenas havia lido isto em um livro que ela pegou na biblioteca da faculdade. ``Por que eu estou lembrando disso?''

Ela simplesmente estava se balançando no ritmo da música e se pegou sendo observada por alguém que estava vestindo roupas normais. Ela sorriu para si mesma. Só Deus sabe o quanto Falls detesta cantadas baratas, mas ela adora receber um elogio ou ver que estão olhando para ela. Não é uma coisa egoísta e narcisista. Bem, na verdade talvez seja. Mas ela gosta de sentir sendo querida, sendo desejada. Falls sabia exatamente qual era o seu problema. Ela era uma romântica em um mundo onde o último romântico já deve ter morrido graças a alguma doença venérea. Já quebrou muito a cara por se deixar envolver demais com sujeitos que não queriam envolvimento nenhum. Onde ela poderia comprar uma borracha, dessas, que apagam o passado?

Ela sempre teve uma queda por rapazes altos, fortes e simpáticos. Daqueles com costas largas e braços enormes. Que garota não tem? Esse que a olhava agora não era nenhum pouco desse jeito. Parecia meio desajeitado. Mas parecia ser um cara legal.

--- O quê? --- Falls perguntou tentando fazer o primeiro contato.

--- Dez e trinta e três --- ele disse bem rápido\mudanca{,} antes de sair andando. Estranho. Ficou pensando um pouco e resolveu procurar Regina. Ela ia procurando o rosto daquele sujeito por onde passava\mudanca{,} no meio daquele mar de pessoas, entre garotos testosterona e garotas arlequinas, \mudanca{sem} sucesso. Ele parecia um cara legal. Era tudo o que Falls precisava: um cara legal.

Pensando nisso ela passava por diversos casais se beijando. É impossível afirmar até que ponto eles podem ser chamados de casais já que são uma união extremamente instável. A meia-vida de muitos é cinco minutos, talvez menos. E relações duradouras? Talvez as pessoas simplesmente não queiram relações de longo prazo. Dá um trabalhão. Ou apenas tenham medo. O dilema do porco-espinho. Ou um porco-espinho enfrenta o frio sozinho, se encolhendo todo\mudanca{,} ou ele se junta a outro para que o calor dos corpos de um e de outro os aqueçam de forma melhor. Mas não é tão simples. Cada vez que um porco espinho se aproxima mais do outro, ambos se ferem. Eles se protegem do frio da solidão, mas se machucam por se aproximarem. Com as pessoas é exatamente assim. Mas como saber o que é certo? Sozinho no inverno é quando mais se deseja a aproximação. Uma vez conquistada, você não pensa em outra coisa a não ser sair dali. A dor da solidão ou a dor de uma relação? Escolha o que escolher, você se arrependerá, dizia o velho Sócrates. A dor incomoda, a quietude perturba.

Ao ver aqueles casais\mudanca{,} Falls pensa na infeliz coincidência muio bem descrita por Dorothy Parker: enquanto a garota jura que é dele, e treme, e suspira leniente, e ele declara que infinita e imortal é a paixão que sente --- Lady, faça uma nota disto: --- Um de vocês mente. Gafanhotos são maus.

Por fim, após rodar mais um pouco pelo local, passando por mais uniões instáveis, ela encontrou Regina\mudanca{,} que lhe apresentou um colega de trabalho dela que as levariam para casa, já que Regina havia exagerado nas bebidas.

--- Como ele é um cavalheiro ele se ofereceu para dirigir por nós --- Regina falou com a voz pastosa, pesada e olhos vermelhos. Tentando controlar uma euforia que teimava em estampar em seu rosto.

--- Segurança em primeiro lugar --- disse o sujeito. --- Sorte que eu não bebo.

Ao se dirigirem à saída, Falls viu um sujeito apertando as nádegas de uma garota que parecia adorar aquilo. O sujeito falou que era uma tal de Julia. Regina fez um comentário censurando o comportamento da mesma. Ainda que por razões diferentes, Falls acabava concordando com ela.

``Oh, humanidade'', pensou Falls. Gafanhotos são maus. Segundo a princesa Atta, ``eles vêm, comem e vão embora''.

Quando ele as estava levando para suas casas, Falls se sentindo incomodada pelo silêncio resolveu puxar assunto:

--- Minha pernas estão doendo muito.

--- As minhas também --- Regina disse como se repreendesse o sujeito que dirigia.

--- Você dançou? --- Falls questionou, surpresa por cogitar a ideia de que Regina dançasse.

--- Claro que não --- Regina respondeu. --- Eu mantenho a postura.

Falls sentiu-se repreendida por Regina. Mas ficou imaginando por que as pernas dela estavam doendo. Ela que não iria perguntar. Ela olhou para Regina e pensou que por mais que ela tivesse suas opiniões e desprezasse qualquer outra e fizesse questão de deixar isso bem claro, ela era uma boa pessoa. Talvez eles pudessem até ter uma conversa normal.

--- Vocês acreditam em amor à primeira-vista? --- Falls perguntou com o olhar perdido na janela.

--- Você está apaixonada? --- Regina perguntou\mudanca{,} como se Falls fosse alguma espécie de idiota.

--- Não. Só por perguntar --- Falls disse meio arrependida.

O sujeito soltou um som de desdém. E olhando para trás disse:

--- Tshh --- O sujeito soltou olhando para trás. --- Não existe amor à primeira vista. Só desejo à primeira vista.

Talvez ele estivesse certo.

--- Então você está fantasiado de pastor? --- Falls perguntou. O estranho se limitou a soltar outro som de desdém pela pergunta.

O estranho apenas disse:

--- Tshh\ldots\,Claro que não.

Falls se sentiu hostilizada pelo sujeito e resolveu provocá-los.

--- E vocês se deram bem na festa?

Os dois tiveram uma estranha troca de olhares. Como de dois culpados conversando com o olhar.

--- NÃO! --- eles exclamaram em sincronia perfeita. Se tivessem combinado dificilmente conseguiriam tal feito.

--- E você? --- perguntou o sujeito olhando pelo espelho retrovisor.

Falls não respondeu logo, ficou olhando para fora e pensando consigo mesma, e então disse:

--- Só um\ldots\,Garoto
